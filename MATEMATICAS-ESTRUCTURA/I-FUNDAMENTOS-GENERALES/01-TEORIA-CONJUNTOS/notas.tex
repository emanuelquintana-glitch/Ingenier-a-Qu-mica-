% Options for packages loaded elsewhere
\PassOptionsToPackage{unicode}{hyperref}
\PassOptionsToPackage{hyphens}{url}
\PassOptionsToPackage{dvipsnames,svgnames,x11names}{xcolor}
%
\documentclass[
  twocolumn,
  10pt]{article}

\usepackage{amsmath,amssymb}
\usepackage{iftex}
\ifPDFTeX
  \usepackage[T1]{fontenc}
  \usepackage[utf8]{inputenc}
  \usepackage{textcomp} % provide euro and other symbols
\else % if luatex or xetex
  \usepackage{unicode-math}
  \defaultfontfeatures{Scale=MatchLowercase}
  \defaultfontfeatures[\rmfamily]{Ligatures=TeX,Scale=1}
\fi
\usepackage[]{libertinus}
\ifPDFTeX\else  
    % xetex/luatex font selection
\fi
% Use upquote if available, for straight quotes in verbatim environments
\IfFileExists{upquote.sty}{\usepackage{upquote}}{}
\IfFileExists{microtype.sty}{% use microtype if available
  \usepackage[]{microtype}
  \UseMicrotypeSet[protrusion]{basicmath} % disable protrusion for tt fonts
}{}
\makeatletter
\@ifundefined{KOMAClassName}{% if non-KOMA class
  \IfFileExists{parskip.sty}{%
    \usepackage{parskip}
  }{% else
    \setlength{\parindent}{0pt}
    \setlength{\parskip}{6pt plus 2pt minus 1pt}}
}{% if KOMA class
  \KOMAoptions{parskip=half}}
\makeatother
\usepackage{xcolor}
\usepackage[margin=1.5cm,top=2.5cm,bottom=2cm]{geometry}
\setlength{\emergencystretch}{3em} % prevent overfull lines
\setcounter{secnumdepth}{5}
% Make \paragraph and \subparagraph free-standing
\ifx\paragraph\undefined\else
  \let\oldparagraph\paragraph
  \renewcommand{\paragraph}[1]{\oldparagraph{#1}\mbox{}}
\fi
\ifx\subparagraph\undefined\else
  \let\oldsubparagraph\subparagraph
  \renewcommand{\subparagraph}[1]{\oldsubparagraph{#1}\mbox{}}
\fi

\usepackage{color}
\usepackage{fancyvrb}
\newcommand{\VerbBar}{|}
\newcommand{\VERB}{\Verb[commandchars=\\\{\}]}
\DefineVerbatimEnvironment{Highlighting}{Verbatim}{commandchars=\\\{\}}
% Add ',fontsize=\small' for more characters per line
\usepackage{framed}
\definecolor{shadecolor}{RGB}{241,243,245}
\newenvironment{Shaded}{\begin{snugshade}}{\end{snugshade}}
\newcommand{\AlertTok}[1]{\textcolor[rgb]{0.68,0.00,0.00}{#1}}
\newcommand{\AnnotationTok}[1]{\textcolor[rgb]{0.37,0.37,0.37}{#1}}
\newcommand{\AttributeTok}[1]{\textcolor[rgb]{0.40,0.45,0.13}{#1}}
\newcommand{\BaseNTok}[1]{\textcolor[rgb]{0.68,0.00,0.00}{#1}}
\newcommand{\BuiltInTok}[1]{\textcolor[rgb]{0.00,0.23,0.31}{#1}}
\newcommand{\CharTok}[1]{\textcolor[rgb]{0.13,0.47,0.30}{#1}}
\newcommand{\CommentTok}[1]{\textcolor[rgb]{0.37,0.37,0.37}{#1}}
\newcommand{\CommentVarTok}[1]{\textcolor[rgb]{0.37,0.37,0.37}{\textit{#1}}}
\newcommand{\ConstantTok}[1]{\textcolor[rgb]{0.56,0.35,0.01}{#1}}
\newcommand{\ControlFlowTok}[1]{\textcolor[rgb]{0.00,0.23,0.31}{#1}}
\newcommand{\DataTypeTok}[1]{\textcolor[rgb]{0.68,0.00,0.00}{#1}}
\newcommand{\DecValTok}[1]{\textcolor[rgb]{0.68,0.00,0.00}{#1}}
\newcommand{\DocumentationTok}[1]{\textcolor[rgb]{0.37,0.37,0.37}{\textit{#1}}}
\newcommand{\ErrorTok}[1]{\textcolor[rgb]{0.68,0.00,0.00}{#1}}
\newcommand{\ExtensionTok}[1]{\textcolor[rgb]{0.00,0.23,0.31}{#1}}
\newcommand{\FloatTok}[1]{\textcolor[rgb]{0.68,0.00,0.00}{#1}}
\newcommand{\FunctionTok}[1]{\textcolor[rgb]{0.28,0.35,0.67}{#1}}
\newcommand{\ImportTok}[1]{\textcolor[rgb]{0.00,0.46,0.62}{#1}}
\newcommand{\InformationTok}[1]{\textcolor[rgb]{0.37,0.37,0.37}{#1}}
\newcommand{\KeywordTok}[1]{\textcolor[rgb]{0.00,0.23,0.31}{#1}}
\newcommand{\NormalTok}[1]{\textcolor[rgb]{0.00,0.23,0.31}{#1}}
\newcommand{\OperatorTok}[1]{\textcolor[rgb]{0.37,0.37,0.37}{#1}}
\newcommand{\OtherTok}[1]{\textcolor[rgb]{0.00,0.23,0.31}{#1}}
\newcommand{\PreprocessorTok}[1]{\textcolor[rgb]{0.68,0.00,0.00}{#1}}
\newcommand{\RegionMarkerTok}[1]{\textcolor[rgb]{0.00,0.23,0.31}{#1}}
\newcommand{\SpecialCharTok}[1]{\textcolor[rgb]{0.37,0.37,0.37}{#1}}
\newcommand{\SpecialStringTok}[1]{\textcolor[rgb]{0.13,0.47,0.30}{#1}}
\newcommand{\StringTok}[1]{\textcolor[rgb]{0.13,0.47,0.30}{#1}}
\newcommand{\VariableTok}[1]{\textcolor[rgb]{0.07,0.07,0.07}{#1}}
\newcommand{\VerbatimStringTok}[1]{\textcolor[rgb]{0.13,0.47,0.30}{#1}}
\newcommand{\WarningTok}[1]{\textcolor[rgb]{0.37,0.37,0.37}{\textit{#1}}}

\providecommand{\tightlist}{%
  \setlength{\itemsep}{0pt}\setlength{\parskip}{0pt}}\usepackage{longtable,booktabs,array}
\usepackage{calc} % for calculating minipage widths
% Correct order of tables after \paragraph or \subparagraph
\usepackage{etoolbox}
\makeatletter
\patchcmd\longtable{\par}{\if@noskipsec\mbox{}\fi\par}{}{}
\makeatother
% Allow footnotes in longtable head/foot
\IfFileExists{footnotehyper.sty}{\usepackage{footnotehyper}}{\usepackage{footnote}}
\makesavenoteenv{longtable}
\usepackage{graphicx}
\makeatletter
\def\maxwidth{\ifdim\Gin@nat@width>\linewidth\linewidth\else\Gin@nat@width\fi}
\def\maxheight{\ifdim\Gin@nat@height>\textheight\textheight\else\Gin@nat@height\fi}
\makeatother
% Scale images if necessary, so that they will not overflow the page
% margins by default, and it is still possible to overwrite the defaults
% using explicit options in \includegraphics[width, height, ...]{}
\setkeys{Gin}{width=\maxwidth,height=\maxheight,keepaspectratio}
% Set default figure placement to htbp
\makeatletter
\def\fps@figure{htbp}
\makeatother
% definitions for citeproc citations
\NewDocumentCommand\citeproctext{}{}
\NewDocumentCommand\citeproc{mm}{%
  \begingroup\def\citeproctext{#2}\cite{#1}\endgroup}
\makeatletter
 % allow citations to break across lines
 \let\@cite@ofmt\@firstofone
 % avoid brackets around text for \cite:
 \def\@biblabel#1{}
 \def\@cite#1#2{{#1\if@tempswa , #2\fi}}
\makeatother
\newlength{\cslhangindent}
\setlength{\cslhangindent}{1.5em}
\newlength{\csllabelwidth}
\setlength{\csllabelwidth}{3em}
\newenvironment{CSLReferences}[2] % #1 hanging-indent, #2 entry-spacing
 {\begin{list}{}{%
  \setlength{\itemindent}{0pt}
  \setlength{\leftmargin}{0pt}
  \setlength{\parsep}{0pt}
  % turn on hanging indent if param 1 is 1
  \ifodd #1
   \setlength{\leftmargin}{\cslhangindent}
   \setlength{\itemindent}{-1\cslhangindent}
  \fi
  % set entry spacing
  \setlength{\itemsep}{#2\baselineskip}}}
 {\end{list}}
\usepackage{calc}
\newcommand{\CSLBlock}[1]{\hfill\break\parbox[t]{\linewidth}{\strut\ignorespaces#1\strut}}
\newcommand{\CSLLeftMargin}[1]{\parbox[t]{\csllabelwidth}{\strut#1\strut}}
\newcommand{\CSLRightInline}[1]{\parbox[t]{\linewidth - \csllabelwidth}{\strut#1\strut}}
\newcommand{\CSLIndent}[1]{\hspace{\cslhangindent}#1}

\makeatletter
\@ifpackageloaded{tcolorbox}{}{\usepackage[skins,breakable]{tcolorbox}}
\@ifpackageloaded{fontawesome5}{}{\usepackage{fontawesome5}}
\definecolor{quarto-callout-color}{HTML}{909090}
\definecolor{quarto-callout-note-color}{HTML}{0758E5}
\definecolor{quarto-callout-important-color}{HTML}{CC1914}
\definecolor{quarto-callout-warning-color}{HTML}{EB9113}
\definecolor{quarto-callout-tip-color}{HTML}{00A047}
\definecolor{quarto-callout-caution-color}{HTML}{FC5300}
\definecolor{quarto-callout-color-frame}{HTML}{acacac}
\definecolor{quarto-callout-note-color-frame}{HTML}{4582ec}
\definecolor{quarto-callout-important-color-frame}{HTML}{d9534f}
\definecolor{quarto-callout-warning-color-frame}{HTML}{f0ad4e}
\definecolor{quarto-callout-tip-color-frame}{HTML}{02b875}
\definecolor{quarto-callout-caution-color-frame}{HTML}{fd7e14}
\makeatother
\makeatletter
\@ifpackageloaded{caption}{}{\usepackage{caption}}
\AtBeginDocument{%
\ifdefined\contentsname
  \renewcommand*\contentsname{Tabla de contenidos}
\else
  \newcommand\contentsname{Tabla de contenidos}
\fi
\ifdefined\listfigurename
  \renewcommand*\listfigurename{Listado de Figuras}
\else
  \newcommand\listfigurename{Listado de Figuras}
\fi
\ifdefined\listtablename
  \renewcommand*\listtablename{Listado de Tablas}
\else
  \newcommand\listtablename{Listado de Tablas}
\fi
\ifdefined\figurename
  \renewcommand*\figurename{Figura}
\else
  \newcommand\figurename{Figura}
\fi
\ifdefined\tablename
  \renewcommand*\tablename{Tabla}
\else
  \newcommand\tablename{Tabla}
\fi
}
\@ifpackageloaded{float}{}{\usepackage{float}}
\floatstyle{ruled}
\@ifundefined{c@chapter}{\newfloat{codelisting}{h}{lop}}{\newfloat{codelisting}{h}{lop}[chapter]}
\floatname{codelisting}{Listado}
\newcommand*\listoflistings{\listof{codelisting}{Listado de Listados}}
\makeatother
\makeatletter
\makeatother
\makeatletter
\@ifpackageloaded{caption}{}{\usepackage{caption}}
\@ifpackageloaded{subcaption}{}{\usepackage{subcaption}}
\makeatother
\ifLuaTeX
\usepackage[bidi=basic]{babel}
\else
\usepackage[bidi=default]{babel}
\fi
\babelprovide[main,import]{spanish}
% get rid of language-specific shorthands (see #6817):
\let\LanguageShortHands\languageshorthands
\def\languageshorthands#1{}
\ifLuaTeX
  \usepackage{selnolig}  % disable illegal ligatures
\fi
\usepackage{bookmark}

\IfFileExists{xurl.sty}{\usepackage{xurl}}{} % add URL line breaks if available
\urlstyle{same} % disable monospaced font for URLs
\hypersetup{
  pdftitle={Fundamentos Matemáticos para Ingenieros},
  pdfauthor={Emanuel Quintana Silva},
  pdflang={es},
  colorlinks=true,
  linkcolor={blue},
  filecolor={Maroon},
  citecolor={Blue},
  urlcolor={Blue},
  pdfcreator={LaTeX via pandoc}}

\title{Fundamentos Matemáticos para Ingenieros}
\usepackage{etoolbox}
\makeatletter
\providecommand{\subtitle}[1]{% add subtitle to \maketitle
  \apptocmd{\@title}{\par {\large #1 \par}}{}{}
}
\makeatother
\subtitle{El lenguaje silencioso que construye la realidad científica}
\author{Emanuel Quintana Silva}
\date{2026-01-01}

\begin{document}
\maketitle

\renewcommand*\contentsname{Tabla de contenidos}
{
\hypersetup{linkcolor=}
\setcounter{tocdepth}{3}
\tableofcontents
}
\begin{tcolorbox}[enhanced jigsaw, leftrule=.75mm, title={📰 THE SCIENCE TIMES \textbar{} INVESTIGACIÓN ESPECIAL}, colback=white, breakable, coltitle=black, bottomrule=.15mm, colframe=quarto-callout-note-color-frame, arc=.35mm, toprule=.15mm, opacitybacktitle=0.6, rightrule=.15mm, opacityback=0, bottomtitle=1mm, toptitle=1mm, colbacktitle=quarto-callout-note-color!10!white, titlerule=0mm, left=2mm]

\textbf{DEEP DIVE} • PAPULA BAND 1 • MATHEMATIK FÜR INGENIEURE • ENERO
2026

\end{tcolorbox}

\section{Introducción: El Átomo de la Lógica
Moderna}\label{introducciuxf3n-el-uxe1tomo-de-la-luxf3gica-moderna}

La transición del pensamiento escolar a la estructura de las disciplinas
científicas superiores requiere más que fórmulas; exige un marco lógico
riguroso que funcione como fundamento inquebrantable. Basado en la obra
monumental de Lothar Papula, \emph{Mathematik für Ingenieure und
Naturwissenschaftler} (Papula, 2024), la teoría de conjuntos se presenta
no como un tema periférico, sino como el \textbf{átomo conceptual} sobre
el cual se construye todo el edificio matemático moderno.

\section{Teoría de Conjuntos: Ontología de la
Unidad}\label{teoruxeda-de-conjuntos-ontologuxeda-de-la-unidad}

\subsection{Definición Fundamental}\label{definiciuxf3n-fundamental}

Un \textbf{conjunto} es la agrupación de objetos bien diferenciados en
una unidad conceptual. Esta definición, aparentemente elemental,
constituye la piedra angular del análisis matemático y establece el
fundamento para todas las estructuras algebraicas superiores utilizadas
en ingeniería y ciencias naturales.

\begin{tcolorbox}[enhanced jigsaw, leftrule=.75mm, title=\textcolor{quarto-callout-note-color}{\faInfo}\hspace{0.5em}{Principio de Pertenencia Absoluta}, colback=white, breakable, coltitle=black, bottomrule=.15mm, colframe=quarto-callout-note-color-frame, arc=.35mm, toprule=.15mm, opacitybacktitle=0.6, rightrule=.15mm, opacityback=0, bottomtitle=1mm, toptitle=1mm, colbacktitle=quarto-callout-note-color!10!white, titlerule=0mm, left=2mm]

Para que un sistema califique como conjunto válido, la relación de
pertenencia debe ser \textbf{absoluta y verificable} sin ambigüedad. No
existen zonas grises: un elemento pertenece (\(\in\)) o es completamente
ajeno (\(\notin\)) al conjunto. Esta dicotomía lógica es fundamental.

\end{tcolorbox}

\subsubsection{Características Estructurales
Esenciales}\label{caracteruxedsticas-estructurales-esenciales}

Los elementos constituyentes de un conjunto deben satisfacer dos
condiciones ineludibles:

\begin{enumerate}
\def\labelenumi{\arabic{enumi}.}
\tightlist
\item
  Ser \textbf{distintos entre sí} (sin duplicados)
\item
  Estar \textbf{bien diferenciados} (sin ambigüedad en su identidad)
\end{enumerate}

Una propiedad crucial es que el \textbf{orden de enumeración carece de
relevancia lógica}, estableciendo así una estructura simétrica
fundamental que distingue a los conjuntos de otras estructuras como las
tuplas ordenadas.

\subsection{Dualidad Representacional}\label{dualidad-representacional}

En el diseño científico riguroso, la claridad de representación es un
valor supremo. Los conjuntos se manifiestan en dos formas
complementarias, cada una optimizada para contextos específicos.

\subsubsection{Método Descriptivo
(Intensional)}\label{muxe9todo-descriptivo-intensional}

Define la esencia del conjunto mediante una propiedad característica.
Este es el método obligatorio para sistemas infinitos donde la
enumeración completa resulta imposible o impráctica.

\[M = \{x \mid x \text{ satisface la propiedad } E \}\]

\textbf{Ejemplo:} El conjunto de números naturales se define como:
\[\mathbb{N} = \{x \mid x \in \mathbb{Z} \text{ y } x \geq 0\}\]

\subsubsection{Método Enumerativo
(Extensional)}\label{muxe9todo-enumerativo-extensional}

Representa el listado directo y explícito de todos los elementos.
Proporciona evidencia tangible y concreta para colecciones finitas.

\[M_3 = \{ -3, -2, -1, 0, 1, 2, 3 \}\]

La enumeración explícita elimina toda ambigüedad interpretativa y
permite verificación inmediata de pertenencia mediante inspección
directa.

\subsection{Operaciones Fundamentales entre
Conjuntos}\label{operaciones-fundamentales-entre-conjuntos}

Las operaciones entre conjuntos (\emph{Mengenoperationen}) constituyen
el álgebra fundamental que permite construir estructuras matemáticas
arbitrariamente complejas a partir de unidades elementales simples.

\subsubsection{Intersección: La Lógica
Conjuntiva}\label{intersecciuxf3n-la-luxf3gica-conjuntiva}

La \textbf{intersección} de dos conjuntos \(A\) y \(B\), simbolizada
como \(A \cap B\), representa el conjunto de elementos que pertenecen
simultáneamente a ambos conjuntos.

\textbf{Definición Formal de Intersección}

\[A \cap B = \{x \mid x \in A \text{ y } x \in B\}\]

La intersección captura la noción lógica del operador ``Y''
(conjunción), siendo fundamental en la resolución de sistemas de
inecuaciones.

\textbf{Aplicación en Sistemas de Inecuaciones}

Considere el sistema: \begin{align}
2x - 4 &> 0 \quad \Rightarrow \quad x > 2 \\
x &< 3
\end{align}

La solución es la intersección de ambos conjuntos solución:
\[L = \{x \mid 2 < x < 3\} = (2,3)\]

Este intervalo abierto representa todos los valores que satisfacen
simultáneamente ambas condiciones.

\begin{Shaded}
\begin{Highlighting}[]
\FunctionTok{library}\NormalTok{(ggplot2)}
\FunctionTok{library}\NormalTok{(ggforce)}

\CommentTok{\# Crear datos para los círculos}
\NormalTok{circulos }\OtherTok{\textless{}{-}} \FunctionTok{data.frame}\NormalTok{(}
  \AttributeTok{x =} \FunctionTok{c}\NormalTok{(}\DecValTok{0}\NormalTok{, }\FloatTok{1.5}\NormalTok{),}
  \AttributeTok{y =} \FunctionTok{c}\NormalTok{(}\DecValTok{0}\NormalTok{, }\DecValTok{0}\NormalTok{),}
  \AttributeTok{r =} \FunctionTok{c}\NormalTok{(}\FloatTok{1.2}\NormalTok{, }\FloatTok{1.2}\NormalTok{),}
  \AttributeTok{conjunto =} \FunctionTok{c}\NormalTok{(}\StringTok{"A"}\NormalTok{, }\StringTok{"B"}\NormalTok{)}
\NormalTok{)}

\FunctionTok{ggplot}\NormalTok{() }\SpecialCharTok{+}
  \FunctionTok{geom\_circle}\NormalTok{(}\AttributeTok{data =}\NormalTok{ circulos, }\FunctionTok{aes}\NormalTok{(}\AttributeTok{x0 =}\NormalTok{ x, }\AttributeTok{y0 =}\NormalTok{ y, }\AttributeTok{r =}\NormalTok{ r, }\AttributeTok{fill =}\NormalTok{ conjunto), }
              \AttributeTok{alpha =} \FloatTok{0.3}\NormalTok{, }\AttributeTok{color =} \StringTok{"black"}\NormalTok{, }\AttributeTok{size =} \FloatTok{1.2}\NormalTok{) }\SpecialCharTok{+}
  \FunctionTok{scale\_fill\_manual}\NormalTok{(}\AttributeTok{values =} \FunctionTok{c}\NormalTok{(}\StringTok{"A"} \OtherTok{=} \StringTok{"\#326891"}\NormalTok{, }\StringTok{"B"} \OtherTok{=} \StringTok{"\#cc0000"}\NormalTok{)) }\SpecialCharTok{+}
  \FunctionTok{annotate}\NormalTok{(}\StringTok{"text"}\NormalTok{, }\AttributeTok{x =} \SpecialCharTok{{-}}\FloatTok{0.6}\NormalTok{, }\AttributeTok{y =} \DecValTok{0}\NormalTok{, }\AttributeTok{label =} \StringTok{"A"}\NormalTok{, }\AttributeTok{size =} \DecValTok{6}\NormalTok{, }\AttributeTok{fontface =} \StringTok{"bold"}\NormalTok{) }\SpecialCharTok{+}
  \FunctionTok{annotate}\NormalTok{(}\StringTok{"text"}\NormalTok{, }\AttributeTok{x =} \FloatTok{2.1}\NormalTok{, }\AttributeTok{y =} \DecValTok{0}\NormalTok{, }\AttributeTok{label =} \StringTok{"B"}\NormalTok{, }\AttributeTok{size =} \DecValTok{6}\NormalTok{, }\AttributeTok{fontface =} \StringTok{"bold"}\NormalTok{) }\SpecialCharTok{+}
  \FunctionTok{annotate}\NormalTok{(}\StringTok{"text"}\NormalTok{, }\AttributeTok{x =} \FloatTok{0.75}\NormalTok{, }\AttributeTok{y =} \DecValTok{0}\NormalTok{, }\AttributeTok{label =} \StringTok{"A ∩ B"}\NormalTok{, }\AttributeTok{size =} \DecValTok{4}\NormalTok{, }\AttributeTok{color =} \StringTok{"\#008060"}\NormalTok{) }\SpecialCharTok{+}
  \FunctionTok{coord\_fixed}\NormalTok{() }\SpecialCharTok{+}
  \FunctionTok{theme\_void}\NormalTok{() }\SpecialCharTok{+}
  \FunctionTok{theme}\NormalTok{(}\AttributeTok{legend.position =} \StringTok{"none"}\NormalTok{,}
        \AttributeTok{plot.title =} \FunctionTok{element\_text}\NormalTok{(}\AttributeTok{hjust =} \FloatTok{0.5}\NormalTok{, }\AttributeTok{face =} \StringTok{"bold"}\NormalTok{)) }\SpecialCharTok{+}
  \FunctionTok{labs}\NormalTok{(}\AttributeTok{title =} \StringTok{"Intersección A ∩ B"}\NormalTok{)}
\end{Highlighting}
\end{Shaded}

\begin{figure}[H]

{\centering \includegraphics{notas_files/figure-pdf/unnamed-chunk-2-1.pdf}

}

\caption{Diagrama de Venn: Intersección de conjuntos A y B}

\end{figure}%

\subsubsection{Unión: La Lógica Disyuntiva
Inclusiva}\label{uniuxf3n-la-luxf3gica-disyuntiva-inclusiva}

La \textbf{unión} de conjuntos \(A\) y \(B\), denotada \(A \cup B\),
agrupa todos los elementos que pertenecen al menos a uno de los
conjuntos.

\textbf{Definición Formal de Unión}

\[A \cup B = \{x \mid x \in A \text{ o } x \in B\}\]

El operador ``o'' es \textbf{inclusivo}: los elementos comunes a ambos
conjuntos forman parte integral de la unión, a diferencia de la
disyunción exclusiva.

\textbf{Ejemplo con Conjuntos Finitos}

Sean \(A = \{1, 2, 3, 4\}\) y \(B = \{1, 5, 6, 7\}\). La unión produce:
\[A \cup B = \{1, 2, 3, 4, 5, 6, 7\}\]

Observe que el elemento común (1) aparece una sola vez en la unión.

\textbf{Unión de Intervalos Continuos}

Para \(M_1 = \{x \mid 0 \le x \le 1\}\) y
\(M_2 = \{x \mid 1 \le x \le 5\}\):
\[M_1 \cup M_2 = \{x \mid 0 \le x \le 5\} = [0,5]\]

Los intervalos se fusionan en uno continuo al compartir el punto
frontera \(x=1\).

\subsubsection{Diferencia: El Operador de
Exclusión}\label{diferencia-el-operador-de-exclusiuxf3n}

La \textbf{diferencia} \(A \setminus B\) (léase ``\(A\) menos \(B\)'')
contiene exclusivamente los elementos de \(A\) que no pertenecen a
\(B\).

\textbf{Definición Formal de Diferencia}

\[A \setminus B = \{x \mid x \in A \text{ y } x \notin B\}\]

\textbf{Construcción de \(\mathbb{N}^*\)}

Los números naturales positivos se definen mediante diferencia:
\[\mathbb{N}^* = \mathbb{N} \setminus \{0\} = \{1, 2, 3, 4, \dots\}\]

\textbf{Diferencia con Conjuntos Finitos}

Si \(A = \{1, 5, 7, 10\}\) y \(B = \{0, 1, 7, 15\}\), entonces:
\[A \setminus B = \{5, 10\}\]

Solo permanecen los elementos exclusivos de \(A\).

\subsection{Relaciones de Inclusión y
Jerarquía}\label{relaciones-de-inclusiuxf3n-y-jerarquuxeda}

Un conjunto \(A\) es \textbf{subconjunto} de \(B\) (denotado
\(A \subseteq B\)) si todo elemento de \(A\) pertenece también a \(B\).

\subsubsection{Identidad y el Conjunto
Vacío}\label{identidad-y-el-conjunto-vacuxedo}

Dos conjuntos \(A\) y \(B\) son \textbf{iguales} (\(A=B\)) si y solo si
contienen exactamente los mismos elementos.

El \textbf{conjunto vacío} (\(\emptyset\)) carece de elementos pero es,
paradójicamente, el concepto más fundamental: permite definir sistemas
sin soluciones y garantiza que la estructura lógica nunca colapse. Es
subconjunto de todo conjunto.

\section{Los Números Reales: Base del
Análisis}\label{los-nuxfameros-reales-base-del-anuxe1lisis}

El conjunto \(\mathbb{R}\) de números reales constituye la columna
vertebral del análisis matemático aplicado, representando la completitud
numérica absoluta necesaria para modelar fenómenos continuos.

\subsection{Estructura Tripartita}\label{estructura-tripartita}

Los números reales incluyen tres categorías exhaustivas:

\begin{enumerate}
\def\labelenumi{\arabic{enumi}.}
\tightlist
\item
  \textbf{Decimales finitos:} Incluyen todos los enteros y fracciones
  que terminan (\(\frac{3}{4} = 0.75\))
\item
  \textbf{Decimales infinitos periódicos:} Números racionales cuya
  expansión decimal se repite (\(\frac{1}{3} = 0.\overline{3}\))
\item
  \textbf{Decimales infinitos no periódicos:} Números irracionales como
  \(\sqrt{2}\), \(\pi\), \(e\)
\end{enumerate}

\begin{tcolorbox}[enhanced jigsaw, leftrule=.75mm, title=\textcolor{quarto-callout-tip-color}{\faLightbulb}\hspace{0.5em}{Correspondencia Biunívoca}, colback=white, breakable, coltitle=black, bottomrule=.15mm, colframe=quarto-callout-tip-color-frame, arc=.35mm, toprule=.15mm, opacitybacktitle=0.6, rightrule=.15mm, opacityback=0, bottomtitle=1mm, toptitle=1mm, colbacktitle=quarto-callout-tip-color!10!white, titlerule=0mm, left=2mm]

Existe una relación uno a uno perfecta entre \(\mathbb{R}\) y los puntos
de la recta numérica dirigida (\emph{Zahlengerade}), estableciendo una
\textbf{geometrización completa} de la aritmética donde cada número
tiene una posición única y cada posición representa un único número.

\end{tcolorbox}

\begin{Shaded}
\begin{Highlighting}[]
\FunctionTok{library}\NormalTok{(ggplot2)}

\CommentTok{\# Crear la recta numérica}
\NormalTok{datos\_puntos }\OtherTok{\textless{}{-}} \FunctionTok{data.frame}\NormalTok{(}
  \AttributeTok{x =} \FunctionTok{c}\NormalTok{(}\SpecialCharTok{{-}}\FloatTok{2.718}\NormalTok{, }\SpecialCharTok{{-}}\DecValTok{2}\NormalTok{, }\SpecialCharTok{{-}}\DecValTok{1}\NormalTok{, }\DecValTok{0}\NormalTok{, }\DecValTok{1}\NormalTok{, }\FloatTok{1.414}\NormalTok{, }\DecValTok{2}\NormalTok{, }\FloatTok{2.5}\NormalTok{, }\DecValTok{3}\NormalTok{),}
  \AttributeTok{y =} \FunctionTok{rep}\NormalTok{(}\DecValTok{0}\NormalTok{, }\DecValTok{9}\NormalTok{),}
  \AttributeTok{etiqueta =} \FunctionTok{c}\NormalTok{(}\StringTok{"{-}e"}\NormalTok{, }\StringTok{"{-}2"}\NormalTok{, }\StringTok{"{-}1"}\NormalTok{, }\StringTok{"0"}\NormalTok{, }\StringTok{"1"}\NormalTok{, }\StringTok{"√2"}\NormalTok{, }\StringTok{"2"}\NormalTok{, }\StringTok{"5/2"}\NormalTok{, }\StringTok{"3"}\NormalTok{),}
  \AttributeTok{tipo =} \FunctionTok{c}\NormalTok{(}\StringTok{"Irracional"}\NormalTok{, }\StringTok{"Entero"}\NormalTok{, }\StringTok{"Entero"}\NormalTok{, }\StringTok{"Entero"}\NormalTok{, }\StringTok{"Entero"}\NormalTok{, }
           \StringTok{"Irracional"}\NormalTok{, }\StringTok{"Entero"}\NormalTok{, }\StringTok{"Racional"}\NormalTok{, }\StringTok{"Entero"}\NormalTok{)}
\NormalTok{)}

\FunctionTok{ggplot}\NormalTok{(datos\_puntos, }\FunctionTok{aes}\NormalTok{(}\AttributeTok{x =}\NormalTok{ x, }\AttributeTok{y =}\NormalTok{ y)) }\SpecialCharTok{+}
  \FunctionTok{geom\_hline}\NormalTok{(}\AttributeTok{yintercept =} \DecValTok{0}\NormalTok{, }\AttributeTok{size =} \FloatTok{1.5}\NormalTok{, }\AttributeTok{arrow =} \FunctionTok{arrow}\NormalTok{(}\AttributeTok{length =} \FunctionTok{unit}\NormalTok{(}\FloatTok{0.3}\NormalTok{, }\StringTok{"cm"}\NormalTok{))) }\SpecialCharTok{+}
  \FunctionTok{geom\_point}\NormalTok{(}\FunctionTok{aes}\NormalTok{(}\AttributeTok{color =}\NormalTok{ tipo), }\AttributeTok{size =} \DecValTok{4}\NormalTok{) }\SpecialCharTok{+}
  \FunctionTok{geom\_text}\NormalTok{(}\FunctionTok{aes}\NormalTok{(}\AttributeTok{label =}\NormalTok{ etiqueta, }\AttributeTok{color =}\NormalTok{ tipo), }\AttributeTok{vjust =} \SpecialCharTok{{-}}\FloatTok{1.5}\NormalTok{, }\AttributeTok{size =} \DecValTok{4}\NormalTok{, }\AttributeTok{fontface =} \StringTok{"bold"}\NormalTok{) }\SpecialCharTok{+}
  \FunctionTok{scale\_color\_manual}\NormalTok{(}\AttributeTok{values =} \FunctionTok{c}\NormalTok{(}\StringTok{"Entero"} \OtherTok{=} \StringTok{"\#121212"}\NormalTok{, }
                                \StringTok{"Racional"} \OtherTok{=} \StringTok{"\#008060"}\NormalTok{,}
                                \StringTok{"Irracional"} \OtherTok{=} \StringTok{"\#326891"}\NormalTok{)) }\SpecialCharTok{+}
  \FunctionTok{scale\_x\_continuous}\NormalTok{(}\AttributeTok{breaks =} \SpecialCharTok{{-}}\DecValTok{3}\SpecialCharTok{:}\DecValTok{4}\NormalTok{, }\AttributeTok{limits =} \FunctionTok{c}\NormalTok{(}\SpecialCharTok{{-}}\DecValTok{4}\NormalTok{, }\DecValTok{4}\NormalTok{)) }\SpecialCharTok{+}
  \FunctionTok{theme\_minimal}\NormalTok{() }\SpecialCharTok{+}
  \FunctionTok{theme}\NormalTok{(}\AttributeTok{axis.title =} \FunctionTok{element\_blank}\NormalTok{(),}
        \AttributeTok{axis.text.y =} \FunctionTok{element\_blank}\NormalTok{(),}
        \AttributeTok{axis.ticks.y =} \FunctionTok{element\_blank}\NormalTok{(),}
        \AttributeTok{panel.grid =} \FunctionTok{element\_blank}\NormalTok{(),}
        \AttributeTok{legend.position =} \StringTok{"top"}\NormalTok{,}
        \AttributeTok{legend.title =} \FunctionTok{element\_blank}\NormalTok{()) }\SpecialCharTok{+}
  \FunctionTok{labs}\NormalTok{(}\AttributeTok{title =} \StringTok{"La Recta Numérica Real ℝ"}\NormalTok{)}
\end{Highlighting}
\end{Shaded}

\begin{verbatim}
Warning in geom_hline(yintercept = 0, size = 1.5, arrow = arrow(length =
unit(0.3, : Ignoring unknown parameters: `arrow`
\end{verbatim}

\begin{figure}[H]

{\centering \includegraphics{notas_files/figure-pdf/unnamed-chunk-3-1.pdf}

}

\caption{La recta numérica real con ejemplos de diferentes tipos}

\end{figure}%

\subsection{Operaciones y Clausura}\label{operaciones-y-clausura}

En \(\mathbb{R}\) se definen cuatro operaciones fundamentales: adición,
sustracción, multiplicación y división. Una característica esencial es
la \textbf{clausura}: la suma, diferencia y producto de dos reales
siempre produce otro real.

Para el cociente existe la restricción absoluta de que \textbf{la
división por cero no está definida}.

\subsection{Leyes Algebraicas
Fundamentales}\label{leyes-algebraicas-fundamentales}

El sistema \((\mathbb{R}, +, \cdot)\) está gobernado por tres familias
de leyes:

\textbf{Leyes Conmutativas}

\[a + b = b + a \quad ; \quad a \cdot b = b \cdot a\]

El orden de los operandos no altera el resultado.

\textbf{Leyes Asociativas}

\[a + (b + c) = (a + b) + c\]
\[a \cdot (b \cdot c) = (a \cdot b) \cdot c\]

La agrupación mediante paréntesis no modifica el valor final.

\textbf{Ley Distributiva}

\[a \cdot (b + c) = a \cdot b + a \cdot c\]

Conecta multiplicación con adición, fundamento del álgebra de
polinomios.

\section{Orden, Inecuaciones y Valor
Absoluto}\label{orden-inecuaciones-y-valor-absoluto}

\subsection{Anordnung der Zahlen}\label{anordnung-der-zahlen}

Para cualquier par de números reales \(a, b \in \mathbb{R}\), se cumple
exactamente una de tres relaciones mutuamente excluyentes: \(a < b\),
\(a = b\), o \(a > b\). Esta \textbf{tricotomía} es la esencia del orden
total en \(\mathbb{R}\).

\textbf{Interpretación Geométrica:}

\begin{itemize}
\tightlist
\item
  \(a < b\): El punto \(a\) se sitúa a la \textbf{izquierda} de \(b\) en
  la recta
\item
  \(a > b\): El punto \(a\) se sitúa a la \textbf{derecha} de \(b\)
\item
  \(a = b\): Ambos puntos coinciden espacialmente
\end{itemize}

\subsection{Transformaciones de
Inecuaciones}\label{transformaciones-de-inecuaciones}

Las inecuaciones se resuelven mediante \textbf{transformaciones
equivalentes} que preservan el conjunto solución:

\textbf{Reglas Fundamentales}

\begin{enumerate}
\def\labelenumi{\arabic{enumi}.}
\item
  \textbf{Adición/Sustracción:} Sumar o restar cualquier término
  \(T(x)\) en ambos lados preserva el signo de desigualdad.
\item
  \textbf{Multiplicación por positivos:} Si \(c > 0\):
  \[a < b \Rightarrow ac < bc\]
\item
  \textbf{Multiplicación por negativos:} Si \(c < 0\), el signo se
  invierte: \[a < b \Rightarrow ac > bc\]
\end{enumerate}

\begin{tcolorbox}[enhanced jigsaw, leftrule=.75mm, title=\textcolor{quarto-callout-warning-color}{\faExclamationTriangle}\hspace{0.5em}{Advertencia Crítica}, colback=white, breakable, coltitle=black, bottomrule=.15mm, colframe=quarto-callout-warning-color-frame, arc=.35mm, toprule=.15mm, opacitybacktitle=0.6, rightrule=.15mm, opacityback=0, bottomtitle=1mm, toptitle=1mm, colbacktitle=quarto-callout-warning-color!10!white, titlerule=0mm, left=2mm]

Cuando se multiplica o divide por un término variable \(T(x)\), se debe
realizar \textbf{distinción de casos} (\emph{Fallunterscheidung}) según
el signo de \(T(x)\) en diferentes intervalos.

\end{tcolorbox}

\textbf{Inecuación Racional con Distinción de Casos}

Resolver: \(\frac{2x-1}{x+2} > 3\)

\textbf{Caso 1:} \(x+2 > 0\) (es decir, \(x > -2\))
\[2x-1 > 3(x+2) \Rightarrow x < -7\] Contradicción con \(x > -2\): sin
soluciones válidas.

\textbf{Caso 2:} \(x+2 < 0\) (es decir, \(x < -2\))
\[2x-1 < 3(x+2) \Rightarrow x > -7\] Intersección: \(-7 < x < -2\) ✓

\textbf{Solución final:} \(L = \{x \mid -7 < x < -2\} = (-7,-2)\)

\subsection{Valor Absoluto: Función de
Distancia}\label{valor-absoluto-funciuxf3n-de-distancia}

El \textbf{valor absoluto} \(|a|\) representa la distancia del número
\(a\) al origen en la recta numérica.

\textbf{Definición por Casos}

\[|a| = \begin{cases} 
a & \text{si } a \geq 0 \\
-a & \text{si } a < 0
\end{cases}\]

\textbf{Propiedades Fundamentales:}

\begin{itemize}
\tightlist
\item
  \(|a| \geq 0\) para todo \(a \in \mathbb{R}\)
\item
  \(|x - a|\) mide la distancia entre \(x\) y \(a\)
\item
  \(|a \cdot b| = |a| \cdot |b|\)
\item
  Desigualdad triangular: \(|a + b| \leq |a| + |b|\)
\end{itemize}

\begin{Shaded}
\begin{Highlighting}[]
\FunctionTok{library}\NormalTok{(ggplot2)}

\NormalTok{x }\OtherTok{\textless{}{-}} \FunctionTok{seq}\NormalTok{(}\SpecialCharTok{{-}}\DecValTok{4}\NormalTok{, }\DecValTok{4}\NormalTok{, }\AttributeTok{by =} \FloatTok{0.01}\NormalTok{)}
\NormalTok{datos }\OtherTok{\textless{}{-}} \FunctionTok{data.frame}\NormalTok{(}
  \AttributeTok{x =} \FunctionTok{rep}\NormalTok{(x, }\DecValTok{3}\NormalTok{),}
  \AttributeTok{y =} \FunctionTok{c}\NormalTok{(}\FunctionTok{abs}\NormalTok{(x), }\FunctionTok{abs}\NormalTok{(x }\SpecialCharTok{{-}} \DecValTok{1}\NormalTok{), }\FunctionTok{abs}\NormalTok{(x) }\SpecialCharTok{{-}} \DecValTok{1}\NormalTok{),}
  \AttributeTok{funcion =} \FunctionTok{rep}\NormalTok{(}\FunctionTok{c}\NormalTok{(}\StringTok{"y = |x|"}\NormalTok{, }\StringTok{"y = |x{-}1|"}\NormalTok{, }\StringTok{"y = |x| {-} 1"}\NormalTok{), }\AttributeTok{each =} \FunctionTok{length}\NormalTok{(x))}
\NormalTok{)}

\FunctionTok{ggplot}\NormalTok{(datos, }\FunctionTok{aes}\NormalTok{(}\AttributeTok{x =}\NormalTok{ x, }\AttributeTok{y =}\NormalTok{ y, }\AttributeTok{color =}\NormalTok{ funcion)) }\SpecialCharTok{+}
  \FunctionTok{geom\_line}\NormalTok{(}\AttributeTok{size =} \FloatTok{1.2}\NormalTok{) }\SpecialCharTok{+}
  \FunctionTok{geom\_hline}\NormalTok{(}\AttributeTok{yintercept =} \DecValTok{0}\NormalTok{, }\AttributeTok{linetype =} \StringTok{"dashed"}\NormalTok{, }\AttributeTok{alpha =} \FloatTok{0.5}\NormalTok{) }\SpecialCharTok{+}
  \FunctionTok{geom\_vline}\NormalTok{(}\AttributeTok{xintercept =} \DecValTok{0}\NormalTok{, }\AttributeTok{linetype =} \StringTok{"dashed"}\NormalTok{, }\AttributeTok{alpha =} \FloatTok{0.5}\NormalTok{) }\SpecialCharTok{+}
  \FunctionTok{scale\_color\_manual}\NormalTok{(}\AttributeTok{values =} \FunctionTok{c}\NormalTok{(}\StringTok{"\#326891"}\NormalTok{, }\StringTok{"\#cc0000"}\NormalTok{, }\StringTok{"\#008060"}\NormalTok{)) }\SpecialCharTok{+}
  \FunctionTok{theme\_minimal}\NormalTok{() }\SpecialCharTok{+}
  \FunctionTok{theme}\NormalTok{(}\AttributeTok{legend.position =} \StringTok{"top"}\NormalTok{,}
        \AttributeTok{legend.title =} \FunctionTok{element\_blank}\NormalTok{(),}
        \AttributeTok{panel.grid.minor =} \FunctionTok{element\_blank}\NormalTok{()) }\SpecialCharTok{+}
  \FunctionTok{labs}\NormalTok{(}\AttributeTok{x =} \StringTok{"x"}\NormalTok{, }\AttributeTok{y =} \StringTok{"y"}\NormalTok{,}
       \AttributeTok{title =} \StringTok{"Funciones de Valor Absoluto"}\NormalTok{) }\SpecialCharTok{+}
  \FunctionTok{coord\_cartesian}\NormalTok{(}\AttributeTok{ylim =} \FunctionTok{c}\NormalTok{(}\SpecialCharTok{{-}}\DecValTok{2}\NormalTok{, }\DecValTok{4}\NormalTok{))}
\end{Highlighting}
\end{Shaded}

\begin{figure}[H]

{\centering \includegraphics{notas_files/figure-pdf/unnamed-chunk-4-1.pdf}

}

\caption{Gráficas de funciones con valor absoluto}

\end{figure}%

\subsection{Ecuaciones con Valor
Absoluto}\label{ecuaciones-con-valor-absoluto}

Las ecuaciones que contienen términos con valor absoluto requieren
\textbf{distinción de casos} basada en el signo del argumento.

\textbf{Resolución Exhaustiva por Casos}

Resolver: \(|x + 2| - 2|x - 3| = 4\)

\textbf{Puntos críticos:} \(x = -2\) y \(x = 3\) dividen la recta en
tres intervalos.

\textbf{Caso I} (\(x < -2\)): Ambos argumentos negativos
\[-(x+2) - 2[-(x-3)] = 4\] \[-x-2+2x-6 = 4 \Rightarrow x = 12\]
\(12 \not< -2\): solución falsa (\emph{Scheinlösung}) ✗

\textbf{Caso II} (\(-2 \leq x \leq 3\)): Primer argumento positivo,
segundo negativo \[(x+2) - 2[-(x-3)] = 4\]
\[x+2+2x-6 = 4 \Rightarrow 3x = 8 \Rightarrow x = \frac{8}{3}\]
\(\frac{8}{3} \approx 2.67 \in [-2,3]\): solución válida ✓

\textbf{Caso III} (\(x > 3\)): Ambos argumentos positivos
\[(x+2) - 2(x-3) = 4\] \[x+2-2x+6 = 4 \Rightarrow x = 4\] \(4 > 3\):
solución válida ✓

\textbf{Conjunto solución:} \(L = \left\{\frac{8}{3}, 4\right\}\)

\begin{tcolorbox}[enhanced jigsaw, leftrule=.75mm, title=\textcolor{quarto-callout-warning-color}{\faExclamationTriangle}\hspace{0.5em}{Verificación Obligatoria}, colback=white, breakable, coltitle=black, bottomrule=.15mm, colframe=quarto-callout-warning-color-frame, arc=.35mm, toprule=.15mm, opacitybacktitle=0.6, rightrule=.15mm, opacityback=0, bottomtitle=1mm, toptitle=1mm, colbacktitle=quarto-callout-warning-color!10!white, titlerule=0mm, left=2mm]

Siempre se debe verificar cada solución candidata en la ecuación
original para descartar soluciones falsas generadas por transformaciones
no equivalentes.

\end{tcolorbox}

\section{Teoría Exhaustiva de
Ecuaciones}\label{teoruxeda-exhaustiva-de-ecuaciones}

\subsection{Ecuaciones Lineales}\label{ecuaciones-lineales}

La forma canónica \(ax + b = 0\) con \(a \neq 0\) posee exactamente una
solución: \[x = -\frac{b}{a}\]

Esta solución única refleja la estructura de campo de \(\mathbb{R}\).

\subsection{Ecuaciones Cuadráticas}\label{ecuaciones-cuadruxe1ticas}

La forma general \(ax^2 + bx + c = 0\) se normaliza dividiendo por \(a\)
para obtener \(x^2 + px + q = 0\) donde \(p = b/a\) y \(q = c/a\).

\textbf{Fórmula \(p,q\) (Forma Normalizada)}

\[x_{1,2} = -\frac{p}{2} \pm \sqrt{\left(\frac{p}{2}\right)^2 - q}\]

El \textbf{discriminante} \(D = \left(\frac{p}{2}\right)^2 - q\)
determina completamente la naturaleza de las soluciones:

\begin{longtable}[]{@{}ll@{}}
\toprule\noalign{}
Discriminante & Soluciones \\
\midrule\noalign{}
\endhead
\bottomrule\noalign{}
\endlastfoot
\(D > 0\) & Dos raíces reales distintas \\
\(D = 0\) & Raíz doble (solución única) \\
\(D < 0\) & Sin soluciones reales (soluciones complejas) \\
\end{longtable}

\begin{Shaded}
\begin{Highlighting}[]
\FunctionTok{library}\NormalTok{(ggplot2)}

\NormalTok{x }\OtherTok{\textless{}{-}} \FunctionTok{seq}\NormalTok{(}\SpecialCharTok{{-}}\DecValTok{3}\NormalTok{, }\DecValTok{5}\NormalTok{, }\AttributeTok{by =} \FloatTok{0.01}\NormalTok{)}

\NormalTok{datos }\OtherTok{\textless{}{-}} \FunctionTok{data.frame}\NormalTok{(}
  \AttributeTok{x =} \FunctionTok{rep}\NormalTok{(x, }\DecValTok{3}\NormalTok{),}
  \AttributeTok{y =} \FunctionTok{c}\NormalTok{(}
\NormalTok{    x}\SpecialCharTok{\^{}}\DecValTok{2} \SpecialCharTok{{-}} \DecValTok{4}\SpecialCharTok{*}\NormalTok{x }\SpecialCharTok{+} \DecValTok{3}\NormalTok{,     }\CommentTok{\# D \textgreater{} 0: dos raíces}
\NormalTok{    x}\SpecialCharTok{\^{}}\DecValTok{2} \SpecialCharTok{{-}} \DecValTok{4}\SpecialCharTok{*}\NormalTok{x }\SpecialCharTok{+} \DecValTok{4}\NormalTok{,     }\CommentTok{\# D = 0: raíz doble}
\NormalTok{    x}\SpecialCharTok{\^{}}\DecValTok{2} \SpecialCharTok{{-}} \DecValTok{4}\SpecialCharTok{*}\NormalTok{x }\SpecialCharTok{+} \DecValTok{5}      \CommentTok{\# D \textless{} 0: sin raíces}
\NormalTok{  ),}
  \AttributeTok{caso =} \FunctionTok{rep}\NormalTok{(}\FunctionTok{c}\NormalTok{(}\StringTok{"D \textgreater{} 0 (dos raíces)"}\NormalTok{, }\StringTok{"D = 0 (raíz doble)"}\NormalTok{, }\StringTok{"D \textless{} 0 (sin raíces)"}\NormalTok{), }
             \AttributeTok{each =} \FunctionTok{length}\NormalTok{(x))}
\NormalTok{)}

\FunctionTok{ggplot}\NormalTok{(datos, }\FunctionTok{aes}\NormalTok{(}\AttributeTok{x =}\NormalTok{ x, }\AttributeTok{y =}\NormalTok{ y, }\AttributeTok{color =}\NormalTok{ caso)) }\SpecialCharTok{+}
  \FunctionTok{geom\_line}\NormalTok{(}\AttributeTok{size =} \FloatTok{1.2}\NormalTok{) }\SpecialCharTok{+}
  \FunctionTok{geom\_hline}\NormalTok{(}\AttributeTok{yintercept =} \DecValTok{0}\NormalTok{, }\AttributeTok{linetype =} \StringTok{"dashed"}\NormalTok{, }\AttributeTok{alpha =} \FloatTok{0.5}\NormalTok{) }\SpecialCharTok{+}
  \FunctionTok{scale\_color\_manual}\NormalTok{(}\AttributeTok{values =} \FunctionTok{c}\NormalTok{(}\StringTok{"\#008060"}\NormalTok{, }\StringTok{"\#326891"}\NormalTok{, }\StringTok{"\#cc0000"}\NormalTok{)) }\SpecialCharTok{+}
  \FunctionTok{theme\_minimal}\NormalTok{() }\SpecialCharTok{+}
  \FunctionTok{theme}\NormalTok{(}\AttributeTok{legend.position =} \StringTok{"top"}\NormalTok{,}
        \AttributeTok{legend.title =} \FunctionTok{element\_blank}\NormalTok{()) }\SpecialCharTok{+}
  \FunctionTok{labs}\NormalTok{(}\AttributeTok{x =} \StringTok{"x"}\NormalTok{, }\AttributeTok{y =} \StringTok{"f(x)"}\NormalTok{,}
       \AttributeTok{title =} \StringTok{"Parábolas y el Discriminante"}\NormalTok{) }\SpecialCharTok{+}
  \FunctionTok{coord\_cartesian}\NormalTok{(}\AttributeTok{ylim =} \FunctionTok{c}\NormalTok{(}\SpecialCharTok{{-}}\DecValTok{2}\NormalTok{, }\DecValTok{6}\NormalTok{))}
\end{Highlighting}
\end{Shaded}

\begin{figure}[H]

{\centering \includegraphics{notas_files/figure-pdf/unnamed-chunk-5-1.pdf}

}

\caption{Comportamiento de funciones cuadráticas según el discriminante}

\end{figure}%

\subsection{Ecuaciones de Grado
Superior}\label{ecuaciones-de-grado-superior}

\textbf{Teorema Fundamental del Álgebra}

Una ecuación polinómica de grado \(n\) tiene como máximo \(n\)
soluciones reales (contando multiplicidades).

\subsubsection{Ecuaciones
Bicuadráticas}\label{ecuaciones-bicuadruxe1ticas}

Para ecuaciones del tipo \(ax^4 + bx^2 + c = 0\):

\textbf{Método de Sustitución}

\textbf{Paso 1:} Sustituir \(u = x^2\), obteniendo:
\(au^2 + bu + c = 0\)

\textbf{Paso 2:} Resolver la ecuación cuadrática en \(u\) usando la
fórmula \(p,q\).

\textbf{Paso 3:} Para cada solución positiva \(u_i > 0\), calcular:
\(x = \pm\sqrt{u_i}\)

\textbf{Importante:} Solo las soluciones \(u > 0\) producen valores
reales de \(x\).

\subsubsection{Esquema de Horner}\label{esquema-de-horner}

Es una herramienta fundamental para:

\begin{enumerate}
\def\labelenumi{\arabic{enumi}.}
\tightlist
\item
  Evaluar polinomios de forma eficiente
\item
  Realizar división sintética
\item
  Reducir el grado de una ecuación si se conoce una raíz
\end{enumerate}

Si \(x_1\) es raíz de \(P(x)\), el esquema permite dividir por
\((x - x_1)\) y reducir el grado del polinomio.

\subsection{Ecuaciones Irracionales}\label{ecuaciones-irracionales}

Son aquellas donde la incógnita aparece bajo un signo de raíz.

\begin{tcolorbox}[enhanced jigsaw, leftrule=.75mm, title=\textcolor{quarto-callout-warning-color}{\faExclamationTriangle}\hspace{0.5em}{Advertencia Crítica}, colback=white, breakable, coltitle=black, bottomrule=.15mm, colframe=quarto-callout-warning-color-frame, arc=.35mm, toprule=.15mm, opacitybacktitle=0.6, rightrule=.15mm, opacityback=0, bottomtitle=1mm, toptitle=1mm, colbacktitle=quarto-callout-warning-color!10!white, titlerule=0mm, left=2mm]

Elevar al cuadrado \textbf{NO} es una transformación equivalente. Pueden
aparecer \textbf{soluciones falsas} (\emph{Scheinlösungen}).

La verificación en la ecuación original es \textbf{OBLIGATORIA}.

\end{tcolorbox}

\textbf{Método de resolución:}

\begin{enumerate}
\def\labelenumi{\arabic{enumi}.}
\tightlist
\item
  Aislar el término con la raíz
\item
  Elevar al cuadrado (o a la potencia correspondiente)
\item
  Resolver la ecuación resultante
\item
  \textbf{Verificar todas las soluciones} en la ecuación original
\end{enumerate}

\section{Sistemas de Ecuaciones
Lineales}\label{sistemas-de-ecuaciones-lineales}

Un sistema de \(m\) ecuaciones con \(n\) incógnitas se puede representar
de forma matricial como:

\[A\mathbf{x} = \mathbf{c}\]

\subsection{Representación Matricial
Completa}\label{representaciuxf3n-matricial-completa}

\textbf{Componentes del Sistema}

\textbf{Matriz de coeficientes:} \[A = \begin{pmatrix} 
a_{11} & a_{12} & \cdots & a_{1n} \\ 
a_{21} & a_{22} & \cdots & a_{2n} \\
\vdots & \vdots & \ddots & \vdots \\
a_{m1} & a_{m2} & \cdots & a_{mn}
\end{pmatrix}\]

\textbf{Vector solución:}
\[\mathbf{x} = \begin{pmatrix} x_1 \\ x_2 \\ \vdots \\ x_n \end{pmatrix}\]

\textbf{Vector de términos constantes:}
\[\mathbf{c} = \begin{pmatrix} c_1 \\ c_2 \\ \vdots \\ c_m \end{pmatrix}\]

\textbf{Matriz ampliada:} \[(A|\mathbf{c}) = \begin{pmatrix} 
a_{11} & \cdots & a_{1n} & | & c_1 \\ 
\vdots & \ddots & \vdots & | & \vdots \\
a_{m1} & \cdots & a_{mn} & | & c_m
\end{pmatrix}\]

\subsection{Algoritmo de Gauss}\label{algoritmo-de-gauss}

El método de eliminación de Gauss transforma el sistema en forma
escalonada mediante \textbf{transformaciones equivalentes}:

\begin{enumerate}
\def\labelenumi{\arabic{enumi}.}
\tightlist
\item
  \textbf{Intercambiar} dos ecuaciones entre sí
\item
  \textbf{Multiplicar} una ecuación por un número distinto de cero
\item
  \textbf{Sumar} a una ecuación un múltiplo de otra
\end{enumerate}

\textbf{Objetivo:} Alcanzar una \textbf{forma trapezoidal} donde todos
los elementos bajo la diagonal principal sean cero.

\begin{Shaded}
\begin{Highlighting}[]
\CommentTok{\# Ejemplo: Resolver sistema 3x3 usando eliminación de Gauss}
\CommentTok{\# Sistema:}
\CommentTok{\#   x + 2y +  z = 9}
\CommentTok{\#  2x {-}  y + 3z = 8}
\CommentTok{\#  3x + 2y + 4z = 15}

\CommentTok{\# Matriz ampliada}
\NormalTok{A\_ampliada }\OtherTok{\textless{}{-}} \FunctionTok{matrix}\NormalTok{(}\FunctionTok{c}\NormalTok{(}
  \DecValTok{1}\NormalTok{,  }\DecValTok{2}\NormalTok{,  }\DecValTok{1}\NormalTok{,  }\DecValTok{9}\NormalTok{,}
  \DecValTok{2}\NormalTok{, }\SpecialCharTok{{-}}\DecValTok{1}\NormalTok{,  }\DecValTok{3}\NormalTok{,  }\DecValTok{8}\NormalTok{,}
  \DecValTok{3}\NormalTok{,  }\DecValTok{2}\NormalTok{,  }\DecValTok{4}\NormalTok{, }\DecValTok{15}
\NormalTok{), }\AttributeTok{nrow =} \DecValTok{3}\NormalTok{, }\AttributeTok{byrow =} \ConstantTok{TRUE}\NormalTok{)}

\FunctionTok{cat}\NormalTok{(}\StringTok{"Matriz ampliada inicial (A|c):}\SpecialCharTok{\textbackslash{}n}\StringTok{"}\NormalTok{)}
\end{Highlighting}
\end{Shaded}

\begin{verbatim}
Matriz ampliada inicial (A|c):
\end{verbatim}

\begin{Shaded}
\begin{Highlighting}[]
\FunctionTok{print}\NormalTok{(A\_ampliada)}
\end{Highlighting}
\end{Shaded}

\begin{verbatim}
     [,1] [,2] [,3] [,4]
[1,]    1    2    1    9
[2,]    2   -1    3    8
[3,]    3    2    4   15
\end{verbatim}

\begin{Shaded}
\begin{Highlighting}[]
\CommentTok{\# Paso 1: Eliminar x en filas 2 y 3}
\NormalTok{A\_ampliada[}\DecValTok{2}\NormalTok{,] }\OtherTok{\textless{}{-}}\NormalTok{ A\_ampliada[}\DecValTok{2}\NormalTok{,] }\SpecialCharTok{{-}} \DecValTok{2} \SpecialCharTok{*}\NormalTok{ A\_ampliada[}\DecValTok{1}\NormalTok{,]}
\NormalTok{A\_ampliada[}\DecValTok{3}\NormalTok{,] }\OtherTok{\textless{}{-}}\NormalTok{ A\_ampliada[}\DecValTok{3}\NormalTok{,] }\SpecialCharTok{{-}} \DecValTok{3} \SpecialCharTok{*}\NormalTok{ A\_ampliada[}\DecValTok{1}\NormalTok{,]}

\FunctionTok{cat}\NormalTok{(}\StringTok{"}\SpecialCharTok{\textbackslash{}n}\StringTok{Después de eliminar x en filas 2 y 3:}\SpecialCharTok{\textbackslash{}n}\StringTok{"}\NormalTok{)}
\end{Highlighting}
\end{Shaded}

\begin{verbatim}

Después de eliminar x en filas 2 y 3:
\end{verbatim}

\begin{Shaded}
\begin{Highlighting}[]
\FunctionTok{print}\NormalTok{(A\_ampliada)}
\end{Highlighting}
\end{Shaded}

\begin{verbatim}
     [,1] [,2] [,3] [,4]
[1,]    1    2    1    9
[2,]    0   -5    1  -10
[3,]    0   -4    1  -12
\end{verbatim}

\begin{Shaded}
\begin{Highlighting}[]
\CommentTok{\# Paso 2: Eliminar y en fila 3}
\NormalTok{A\_ampliada[}\DecValTok{3}\NormalTok{,] }\OtherTok{\textless{}{-}}\NormalTok{ A\_ampliada[}\DecValTok{3}\NormalTok{,] }\SpecialCharTok{{-}}\NormalTok{ (A\_ampliada[}\DecValTok{3}\NormalTok{,}\DecValTok{2}\NormalTok{] }\SpecialCharTok{/}\NormalTok{ A\_ampliada[}\DecValTok{2}\NormalTok{,}\DecValTok{2}\NormalTok{]) }\SpecialCharTok{*}\NormalTok{ A\_ampliada[}\DecValTok{2}\NormalTok{,]}

\FunctionTok{cat}\NormalTok{(}\StringTok{"}\SpecialCharTok{\textbackslash{}n}\StringTok{Forma escalonada (trapezoidal):}\SpecialCharTok{\textbackslash{}n}\StringTok{"}\NormalTok{)}
\end{Highlighting}
\end{Shaded}

\begin{verbatim}

Forma escalonada (trapezoidal):
\end{verbatim}

\begin{Shaded}
\begin{Highlighting}[]
\FunctionTok{print}\NormalTok{(A\_ampliada)}
\end{Highlighting}
\end{Shaded}

\begin{verbatim}
     [,1] [,2] [,3] [,4]
[1,]    1    2  1.0    9
[2,]    0   -5  1.0  -10
[3,]    0    0  0.2   -4
\end{verbatim}

\begin{Shaded}
\begin{Highlighting}[]
\CommentTok{\# Paso 3: Sustitución regresiva}
\NormalTok{z }\OtherTok{\textless{}{-}}\NormalTok{ A\_ampliada[}\DecValTok{3}\NormalTok{,}\DecValTok{4}\NormalTok{] }\SpecialCharTok{/}\NormalTok{ A\_ampliada[}\DecValTok{3}\NormalTok{,}\DecValTok{3}\NormalTok{]}
\NormalTok{y }\OtherTok{\textless{}{-}}\NormalTok{ (A\_ampliada[}\DecValTok{2}\NormalTok{,}\DecValTok{4}\NormalTok{] }\SpecialCharTok{{-}}\NormalTok{ A\_ampliada[}\DecValTok{2}\NormalTok{,}\DecValTok{3}\NormalTok{] }\SpecialCharTok{*}\NormalTok{ z) }\SpecialCharTok{/}\NormalTok{ A\_ampliada[}\DecValTok{2}\NormalTok{,}\DecValTok{2}\NormalTok{]}
\NormalTok{x }\OtherTok{\textless{}{-}}\NormalTok{ (A\_ampliada[}\DecValTok{1}\NormalTok{,}\DecValTok{4}\NormalTok{] }\SpecialCharTok{{-}}\NormalTok{ A\_ampliada[}\DecValTok{1}\NormalTok{,}\DecValTok{3}\NormalTok{] }\SpecialCharTok{*}\NormalTok{ z }\SpecialCharTok{{-}}\NormalTok{ A\_ampliada[}\DecValTok{1}\NormalTok{,}\DecValTok{2}\NormalTok{] }\SpecialCharTok{*}\NormalTok{ y) }\SpecialCharTok{/}\NormalTok{ A\_ampliada[}\DecValTok{1}\NormalTok{,}\DecValTok{1}\NormalTok{]}

\FunctionTok{cat}\NormalTok{(}\StringTok{"}\SpecialCharTok{\textbackslash{}n}\StringTok{Solución del sistema:}\SpecialCharTok{\textbackslash{}n}\StringTok{"}\NormalTok{)}
\end{Highlighting}
\end{Shaded}

\begin{verbatim}

Solución del sistema:
\end{verbatim}

\begin{Shaded}
\begin{Highlighting}[]
\FunctionTok{cat}\NormalTok{(}\FunctionTok{sprintf}\NormalTok{(}\StringTok{"x = \%.4f}\SpecialCharTok{\textbackslash{}n}\StringTok{"}\NormalTok{, x))}
\end{Highlighting}
\end{Shaded}

\begin{verbatim}
x = 33.0000
\end{verbatim}

\begin{Shaded}
\begin{Highlighting}[]
\FunctionTok{cat}\NormalTok{(}\FunctionTok{sprintf}\NormalTok{(}\StringTok{"y = \%.4f}\SpecialCharTok{\textbackslash{}n}\StringTok{"}\NormalTok{, y))}
\end{Highlighting}
\end{Shaded}

\begin{verbatim}
y = -2.0000
\end{verbatim}

\begin{Shaded}
\begin{Highlighting}[]
\FunctionTok{cat}\NormalTok{(}\FunctionTok{sprintf}\NormalTok{(}\StringTok{"z = \%.4f}\SpecialCharTok{\textbackslash{}n}\StringTok{"}\NormalTok{, z))}
\end{Highlighting}
\end{Shaded}

\begin{verbatim}
z = -20.0000
\end{verbatim}

\subsection{Análisis de Solubilidad}\label{anuxe1lisis-de-solubilidad}

\textbf{Criterio Fundamental de Solubilidad}

Un sistema es resoluble (tiene al menos una solución) si y solo si:
\(Rg(A) = Rg(A|\mathbf{c})\)

donde \(Rg\) denota el rango de la matriz.

\textbf{Clasificación según el rango \(r\):}

\begin{longtable}[]{@{}lll@{}}
\caption{Comportamiento de sistemas lineales según el
rango}\label{tbl-sistemas}\tabularnewline
\toprule\noalign{}
Condición & Tipo de Sistema & Soluciones \\
\midrule\noalign{}
\endfirsthead
\toprule\noalign{}
Condición & Tipo de Sistema & Soluciones \\
\midrule\noalign{}
\endhead
\bottomrule\noalign{}
\endlastfoot
\(r = n\) & Compatible determinado & Solución única \\
\(r < n\) & Compatible indeterminado & Infinitas (\(n-r\) parámetros) \\
\(Rg(A) < Rg(A\|\mathbf{c})\) & Incompatible & Sin solución \\
\end{longtable}

\begin{Shaded}
\begin{Highlighting}[]
\FunctionTok{library}\NormalTok{(ggplot2)}
\FunctionTok{library}\NormalTok{(patchwork)}

\NormalTok{x }\OtherTok{\textless{}{-}} \FunctionTok{seq}\NormalTok{(}\SpecialCharTok{{-}}\DecValTok{2}\NormalTok{, }\DecValTok{5}\NormalTok{, }\AttributeTok{by =} \FloatTok{0.01}\NormalTok{)}

\CommentTok{\# Sistema con solución única}
\NormalTok{p1 }\OtherTok{\textless{}{-}} \FunctionTok{ggplot}\NormalTok{() }\SpecialCharTok{+}
  \FunctionTok{geom\_abline}\NormalTok{(}\AttributeTok{slope =} \SpecialCharTok{{-}}\DecValTok{1}\SpecialCharTok{/}\DecValTok{2}\NormalTok{, }\AttributeTok{intercept =} \DecValTok{3}\NormalTok{, }\AttributeTok{color =} \StringTok{"\#326891"}\NormalTok{, }\AttributeTok{size =} \FloatTok{1.2}\NormalTok{) }\SpecialCharTok{+}
  \FunctionTok{geom\_abline}\NormalTok{(}\AttributeTok{slope =} \DecValTok{2}\NormalTok{, }\AttributeTok{intercept =} \SpecialCharTok{{-}}\DecValTok{1}\NormalTok{, }\AttributeTok{color =} \StringTok{"\#cc0000"}\NormalTok{, }\AttributeTok{size =} \FloatTok{1.2}\NormalTok{) }\SpecialCharTok{+}
  \FunctionTok{geom\_point}\NormalTok{(}\FunctionTok{aes}\NormalTok{(}\AttributeTok{x =} \DecValTok{2}\NormalTok{, }\AttributeTok{y =} \DecValTok{2}\NormalTok{), }\AttributeTok{size =} \DecValTok{4}\NormalTok{, }\AttributeTok{color =} \StringTok{"\#008060"}\NormalTok{) }\SpecialCharTok{+}
  \FunctionTok{annotate}\NormalTok{(}\StringTok{"text"}\NormalTok{, }\AttributeTok{x =} \DecValTok{2}\NormalTok{, }\AttributeTok{y =} \FloatTok{2.5}\NormalTok{, }\AttributeTok{label =} \StringTok{"(2, 2)"}\NormalTok{, }\AttributeTok{color =} \StringTok{"\#008060"}\NormalTok{, }\AttributeTok{fontface =} \StringTok{"bold"}\NormalTok{) }\SpecialCharTok{+}
  \FunctionTok{coord\_cartesian}\NormalTok{(}\AttributeTok{xlim =} \FunctionTok{c}\NormalTok{(}\SpecialCharTok{{-}}\DecValTok{1}\NormalTok{, }\DecValTok{5}\NormalTok{), }\AttributeTok{ylim =} \FunctionTok{c}\NormalTok{(}\SpecialCharTok{{-}}\DecValTok{1}\NormalTok{, }\DecValTok{5}\NormalTok{)) }\SpecialCharTok{+}
  \FunctionTok{theme\_minimal}\NormalTok{() }\SpecialCharTok{+}
  \FunctionTok{labs}\NormalTok{(}\AttributeTok{title =} \StringTok{"Solución Única"}\NormalTok{, }\AttributeTok{x =} \StringTok{"x"}\NormalTok{, }\AttributeTok{y =} \StringTok{"y"}\NormalTok{) }\SpecialCharTok{+}
  \FunctionTok{theme}\NormalTok{(}\AttributeTok{plot.title =} \FunctionTok{element\_text}\NormalTok{(}\AttributeTok{face =} \StringTok{"bold"}\NormalTok{, }\AttributeTok{hjust =} \FloatTok{0.5}\NormalTok{))}

\CommentTok{\# Sistema con infinitas soluciones (rectas coincidentes)}
\NormalTok{p2 }\OtherTok{\textless{}{-}} \FunctionTok{ggplot}\NormalTok{() }\SpecialCharTok{+}
  \FunctionTok{geom\_abline}\NormalTok{(}\AttributeTok{slope =} \DecValTok{1}\NormalTok{, }\AttributeTok{intercept =} \DecValTok{1}\NormalTok{, }\AttributeTok{color =} \StringTok{"\#326891"}\NormalTok{, }\AttributeTok{size =} \FloatTok{1.2}\NormalTok{) }\SpecialCharTok{+}
  \FunctionTok{geom\_abline}\NormalTok{(}\AttributeTok{slope =} \DecValTok{1}\NormalTok{, }\AttributeTok{intercept =} \DecValTok{1}\NormalTok{, }\AttributeTok{color =} \StringTok{"\#cc0000"}\NormalTok{, }\AttributeTok{size =} \FloatTok{1.2}\NormalTok{, }\AttributeTok{linetype =} \StringTok{"dashed"}\NormalTok{) }\SpecialCharTok{+}
  \FunctionTok{annotate}\NormalTok{(}\StringTok{"text"}\NormalTok{, }\AttributeTok{x =} \DecValTok{2}\NormalTok{, }\AttributeTok{y =} \FloatTok{3.5}\NormalTok{, }\AttributeTok{label =} \StringTok{"Infinitas soluciones}\SpecialCharTok{\textbackslash{}n}\StringTok{(rectas coincidentes)"}\NormalTok{, }
           \AttributeTok{color =} \StringTok{"\#326891"}\NormalTok{, }\AttributeTok{fontface =} \StringTok{"bold"}\NormalTok{, }\AttributeTok{size =} \DecValTok{3}\NormalTok{) }\SpecialCharTok{+}
  \FunctionTok{coord\_cartesian}\NormalTok{(}\AttributeTok{xlim =} \FunctionTok{c}\NormalTok{(}\SpecialCharTok{{-}}\DecValTok{1}\NormalTok{, }\DecValTok{5}\NormalTok{), }\AttributeTok{ylim =} \FunctionTok{c}\NormalTok{(}\SpecialCharTok{{-}}\DecValTok{1}\NormalTok{, }\DecValTok{5}\NormalTok{)) }\SpecialCharTok{+}
  \FunctionTok{theme\_minimal}\NormalTok{() }\SpecialCharTok{+}
  \FunctionTok{labs}\NormalTok{(}\AttributeTok{title =} \StringTok{"Infinitas Soluciones"}\NormalTok{, }\AttributeTok{x =} \StringTok{"x"}\NormalTok{, }\AttributeTok{y =} \StringTok{"y"}\NormalTok{) }\SpecialCharTok{+}
  \FunctionTok{theme}\NormalTok{(}\AttributeTok{plot.title =} \FunctionTok{element\_text}\NormalTok{(}\AttributeTok{face =} \StringTok{"bold"}\NormalTok{, }\AttributeTok{hjust =} \FloatTok{0.5}\NormalTok{))}

\CommentTok{\# Sistema sin solución (rectas paralelas)}
\NormalTok{p3 }\OtherTok{\textless{}{-}} \FunctionTok{ggplot}\NormalTok{() }\SpecialCharTok{+}
  \FunctionTok{geom\_abline}\NormalTok{(}\AttributeTok{slope =} \DecValTok{1}\NormalTok{, }\AttributeTok{intercept =} \DecValTok{1}\NormalTok{, }\AttributeTok{color =} \StringTok{"\#326891"}\NormalTok{, }\AttributeTok{size =} \FloatTok{1.2}\NormalTok{) }\SpecialCharTok{+}
  \FunctionTok{geom\_abline}\NormalTok{(}\AttributeTok{slope =} \DecValTok{1}\NormalTok{, }\AttributeTok{intercept =} \DecValTok{3}\NormalTok{, }\AttributeTok{color =} \StringTok{"\#cc0000"}\NormalTok{, }\AttributeTok{size =} \FloatTok{1.2}\NormalTok{) }\SpecialCharTok{+}
  \FunctionTok{annotate}\NormalTok{(}\StringTok{"text"}\NormalTok{, }\AttributeTok{x =} \DecValTok{2}\NormalTok{, }\AttributeTok{y =} \FloatTok{4.5}\NormalTok{, }\AttributeTok{label =} \StringTok{"Sin solución}\SpecialCharTok{\textbackslash{}n}\StringTok{(paralelas)"}\NormalTok{, }
           \AttributeTok{color =} \StringTok{"\#cc0000"}\NormalTok{, }\AttributeTok{fontface =} \StringTok{"bold"}\NormalTok{, }\AttributeTok{size =} \DecValTok{3}\NormalTok{) }\SpecialCharTok{+}
  \FunctionTok{coord\_cartesian}\NormalTok{(}\AttributeTok{xlim =} \FunctionTok{c}\NormalTok{(}\SpecialCharTok{{-}}\DecValTok{1}\NormalTok{, }\DecValTok{5}\NormalTok{), }\AttributeTok{ylim =} \FunctionTok{c}\NormalTok{(}\SpecialCharTok{{-}}\DecValTok{1}\NormalTok{, }\DecValTok{5}\NormalTok{)) }\SpecialCharTok{+}
  \FunctionTok{theme\_minimal}\NormalTok{() }\SpecialCharTok{+}
  \FunctionTok{labs}\NormalTok{(}\AttributeTok{title =} \StringTok{"Sin Solución"}\NormalTok{, }\AttributeTok{x =} \StringTok{"x"}\NormalTok{, }\AttributeTok{y =} \StringTok{"y"}\NormalTok{) }\SpecialCharTok{+}
  \FunctionTok{theme}\NormalTok{(}\AttributeTok{plot.title =} \FunctionTok{element\_text}\NormalTok{(}\AttributeTok{face =} \StringTok{"bold"}\NormalTok{, }\AttributeTok{hjust =} \FloatTok{0.5}\NormalTok{))}

\NormalTok{p1 }\SpecialCharTok{+}\NormalTok{ p2 }\SpecialCharTok{+}\NormalTok{ p3}
\end{Highlighting}
\end{Shaded}

\begin{figure}[H]

{\centering \includegraphics{notas_files/figure-pdf/unnamed-chunk-6-1.pdf}

}

\caption{Visualización de sistemas 2x2: único, infinitos y sin solución}

\end{figure}%

\subsection{Regla de Cramer}\label{regla-de-cramer}

Para sistemas cuadrados (\(n \times n\)) con matriz regular
(\(\det(A) \neq 0\)):

\textbf{Fórmula de Cramer}

\(x_i = \frac{\det(A_i)}{\det(A)}\)

donde \(A_i\) es la matriz \(A\) con la columna \(i\) reemplazada por el
vector \(\mathbf{c}\).

\begin{Shaded}
\begin{Highlighting}[]
\CommentTok{\# Ejemplo con Regla de Cramer}
\CommentTok{\# Sistema: 2x + 3y = 8}
\CommentTok{\#          x {-} y = {-}1}

\NormalTok{A }\OtherTok{\textless{}{-}} \FunctionTok{matrix}\NormalTok{(}\FunctionTok{c}\NormalTok{(}\DecValTok{2}\NormalTok{, }\DecValTok{3}\NormalTok{, }\DecValTok{1}\NormalTok{, }\SpecialCharTok{{-}}\DecValTok{1}\NormalTok{), }\AttributeTok{nrow =} \DecValTok{2}\NormalTok{, }\AttributeTok{byrow =} \ConstantTok{TRUE}\NormalTok{)}
\NormalTok{c\_vec }\OtherTok{\textless{}{-}} \FunctionTok{c}\NormalTok{(}\DecValTok{8}\NormalTok{, }\SpecialCharTok{{-}}\DecValTok{1}\NormalTok{)}

\FunctionTok{cat}\NormalTok{(}\StringTok{"Sistema de ecuaciones:}\SpecialCharTok{\textbackslash{}n}\StringTok{"}\NormalTok{)}
\end{Highlighting}
\end{Shaded}

\begin{verbatim}
Sistema de ecuaciones:
\end{verbatim}

\begin{Shaded}
\begin{Highlighting}[]
\FunctionTok{cat}\NormalTok{(}\StringTok{"2x + 3y = 8}\SpecialCharTok{\textbackslash{}n}\StringTok{"}\NormalTok{)}
\end{Highlighting}
\end{Shaded}

\begin{verbatim}
2x + 3y = 8
\end{verbatim}

\begin{Shaded}
\begin{Highlighting}[]
\FunctionTok{cat}\NormalTok{(}\StringTok{" x {-}  y = {-}1}\SpecialCharTok{\textbackslash{}n\textbackslash{}n}\StringTok{"}\NormalTok{)}
\end{Highlighting}
\end{Shaded}

\begin{verbatim}
 x -  y = -1
\end{verbatim}

\begin{Shaded}
\begin{Highlighting}[]
\CommentTok{\# Determinante de A}
\NormalTok{det\_A }\OtherTok{\textless{}{-}} \FunctionTok{det}\NormalTok{(A)}
\FunctionTok{cat}\NormalTok{(}\FunctionTok{sprintf}\NormalTok{(}\StringTok{"det(A) = \%.0f}\SpecialCharTok{\textbackslash{}n\textbackslash{}n}\StringTok{"}\NormalTok{, det\_A))}
\end{Highlighting}
\end{Shaded}

\begin{verbatim}
det(A) = -5
\end{verbatim}

\begin{Shaded}
\begin{Highlighting}[]
\CommentTok{\# A1: reemplazar primera columna con c}
\NormalTok{A1 }\OtherTok{\textless{}{-}}\NormalTok{ A}
\NormalTok{A1[,}\DecValTok{1}\NormalTok{] }\OtherTok{\textless{}{-}}\NormalTok{ c\_vec}
\NormalTok{det\_A1 }\OtherTok{\textless{}{-}} \FunctionTok{det}\NormalTok{(A1)}

\CommentTok{\# A2: reemplazar segunda columna con c}
\NormalTok{A2 }\OtherTok{\textless{}{-}}\NormalTok{ A}
\NormalTok{A2[,}\DecValTok{2}\NormalTok{] }\OtherTok{\textless{}{-}}\NormalTok{ c\_vec}
\NormalTok{det\_A2 }\OtherTok{\textless{}{-}} \FunctionTok{det}\NormalTok{(A2)}

\CommentTok{\# Soluciones}
\NormalTok{x }\OtherTok{\textless{}{-}}\NormalTok{ det\_A1 }\SpecialCharTok{/}\NormalTok{ det\_A}
\NormalTok{y }\OtherTok{\textless{}{-}}\NormalTok{ det\_A2 }\SpecialCharTok{/}\NormalTok{ det\_A}

\FunctionTok{cat}\NormalTok{(}\FunctionTok{sprintf}\NormalTok{(}\StringTok{"x = det(A1)/det(A) = \%.0f/\%.0f = \%.0f}\SpecialCharTok{\textbackslash{}n}\StringTok{"}\NormalTok{, det\_A1, det\_A, x))}
\end{Highlighting}
\end{Shaded}

\begin{verbatim}
x = det(A1)/det(A) = -5/-5 = 1
\end{verbatim}

\begin{Shaded}
\begin{Highlighting}[]
\FunctionTok{cat}\NormalTok{(}\FunctionTok{sprintf}\NormalTok{(}\StringTok{"y = det(A2)/det(A) = \%.0f/\%.0f = \%.0f}\SpecialCharTok{\textbackslash{}n}\StringTok{"}\NormalTok{, det\_A2, det\_A, y))}
\end{Highlighting}
\end{Shaded}

\begin{verbatim}
y = det(A2)/det(A) = -10/-5 = 2
\end{verbatim}

\subsection{Aplicación: Circuitos
Eléctricos}\label{aplicaciuxf3n-circuitos-eluxe9ctricos}

Las \textbf{Reglas de Kirchhoff} permiten analizar redes eléctricas
usando sistemas de ecuaciones lineales.

\textbf{Análisis de Red con Tres Corrientes}

Para un circuito con resistencias \(R_1, R_2, R_3\) y fuente de tensión
\(U\):

\textbf{Regla de los Nudos (\emph{Knotenpunktsregel}):}
\(I_1 - I_2 - I_3 = 0\)

La suma de corrientes entrantes y salientes en un nudo es cero.

\textbf{Reglas de las Mallas (\emph{Maschenregel}):}
\(R_1 I_1 + R_2 I_2 = U\) \(R_2 I_2 - R_3 I_3 = 0\)

En cada malla, la suma de tensiones es cero.

\textbf{Matriz del Sistema:} \(A = \begin{pmatrix}
1 & -1 & -1 \\
R_1 & R_2 & 0 \\
0 & R_2 & -R_3
\end{pmatrix}\)

\begin{Shaded}
\begin{Highlighting}[]
\CommentTok{\# Resolver circuito eléctrico}
\CommentTok{\# Valores: R1 = 2Ω, R2 = 3Ω, R3 = 4Ω, U = 12V}

\NormalTok{R1 }\OtherTok{\textless{}{-}} \DecValTok{2}
\NormalTok{R2 }\OtherTok{\textless{}{-}} \DecValTok{3}
\NormalTok{R3 }\OtherTok{\textless{}{-}} \DecValTok{4}
\NormalTok{U }\OtherTok{\textless{}{-}} \DecValTok{12}

\CommentTok{\# Matriz de coeficientes}
\NormalTok{A\_circuito }\OtherTok{\textless{}{-}} \FunctionTok{matrix}\NormalTok{(}\FunctionTok{c}\NormalTok{(}
  \DecValTok{1}\NormalTok{, }\SpecialCharTok{{-}}\DecValTok{1}\NormalTok{, }\SpecialCharTok{{-}}\DecValTok{1}\NormalTok{,}
\NormalTok{  R1, R2, }\DecValTok{0}\NormalTok{,}
  \DecValTok{0}\NormalTok{, R2, }\SpecialCharTok{{-}}\NormalTok{R3}
\NormalTok{), }\AttributeTok{nrow =} \DecValTok{3}\NormalTok{, }\AttributeTok{byrow =} \ConstantTok{TRUE}\NormalTok{)}

\NormalTok{c\_circuito }\OtherTok{\textless{}{-}} \FunctionTok{c}\NormalTok{(}\DecValTok{0}\NormalTok{, U, }\DecValTok{0}\NormalTok{)}

\FunctionTok{cat}\NormalTok{(}\StringTok{"Sistema del circuito eléctrico:}\SpecialCharTok{\textbackslash{}n}\StringTok{"}\NormalTok{)}
\end{Highlighting}
\end{Shaded}

\begin{verbatim}
Sistema del circuito eléctrico:
\end{verbatim}

\begin{Shaded}
\begin{Highlighting}[]
\FunctionTok{cat}\NormalTok{(}\FunctionTok{sprintf}\NormalTok{(}\StringTok{"I1 {-} I2 {-} I3 = 0}\SpecialCharTok{\textbackslash{}n}\StringTok{"}\NormalTok{))}
\end{Highlighting}
\end{Shaded}

\begin{verbatim}
I1 - I2 - I3 = 0
\end{verbatim}

\begin{Shaded}
\begin{Highlighting}[]
\FunctionTok{cat}\NormalTok{(}\FunctionTok{sprintf}\NormalTok{(}\StringTok{"\%.0f*I1 + \%.0f*I2 = \%.0f}\SpecialCharTok{\textbackslash{}n}\StringTok{"}\NormalTok{, R1, R2, U))}
\end{Highlighting}
\end{Shaded}

\begin{verbatim}
2*I1 + 3*I2 = 12
\end{verbatim}

\begin{Shaded}
\begin{Highlighting}[]
\FunctionTok{cat}\NormalTok{(}\FunctionTok{sprintf}\NormalTok{(}\StringTok{"\%.0f*I2 {-} \%.0f*I3 = 0}\SpecialCharTok{\textbackslash{}n\textbackslash{}n}\StringTok{"}\NormalTok{, R2, R3))}
\end{Highlighting}
\end{Shaded}

\begin{verbatim}
3*I2 - 4*I3 = 0
\end{verbatim}

\begin{Shaded}
\begin{Highlighting}[]
\CommentTok{\# Resolver usando solve()}
\NormalTok{corrientes }\OtherTok{\textless{}{-}} \FunctionTok{solve}\NormalTok{(A\_circuito, c\_circuito)}

\FunctionTok{cat}\NormalTok{(}\StringTok{"Corrientes del circuito:}\SpecialCharTok{\textbackslash{}n}\StringTok{"}\NormalTok{)}
\end{Highlighting}
\end{Shaded}

\begin{verbatim}
Corrientes del circuito:
\end{verbatim}

\begin{Shaded}
\begin{Highlighting}[]
\FunctionTok{cat}\NormalTok{(}\FunctionTok{sprintf}\NormalTok{(}\StringTok{"I1 = \%.4f A}\SpecialCharTok{\textbackslash{}n}\StringTok{"}\NormalTok{, corrientes[}\DecValTok{1}\NormalTok{]))}
\end{Highlighting}
\end{Shaded}

\begin{verbatim}
I1 = 3.2308 A
\end{verbatim}

\begin{Shaded}
\begin{Highlighting}[]
\FunctionTok{cat}\NormalTok{(}\FunctionTok{sprintf}\NormalTok{(}\StringTok{"I2 = \%.4f A}\SpecialCharTok{\textbackslash{}n}\StringTok{"}\NormalTok{, corrientes[}\DecValTok{2}\NormalTok{]))}
\end{Highlighting}
\end{Shaded}

\begin{verbatim}
I2 = 1.8462 A
\end{verbatim}

\begin{Shaded}
\begin{Highlighting}[]
\FunctionTok{cat}\NormalTok{(}\FunctionTok{sprintf}\NormalTok{(}\StringTok{"I3 = \%.4f A}\SpecialCharTok{\textbackslash{}n}\StringTok{"}\NormalTok{, corrientes[}\DecValTok{3}\NormalTok{]))}
\end{Highlighting}
\end{Shaded}

\begin{verbatim}
I3 = 1.3846 A
\end{verbatim}

\begin{Shaded}
\begin{Highlighting}[]
\CommentTok{\# Verificación}
\FunctionTok{cat}\NormalTok{(}\StringTok{"}\SpecialCharTok{\textbackslash{}n}\StringTok{Verificación de las ecuaciones:}\SpecialCharTok{\textbackslash{}n}\StringTok{"}\NormalTok{)}
\end{Highlighting}
\end{Shaded}

\begin{verbatim}

Verificación de las ecuaciones:
\end{verbatim}

\begin{Shaded}
\begin{Highlighting}[]
\FunctionTok{cat}\NormalTok{(}\FunctionTok{sprintf}\NormalTok{(}\StringTok{"I1 {-} I2 {-} I3 = \%.6f (≈ 0) ✓}\SpecialCharTok{\textbackslash{}n}\StringTok{"}\NormalTok{, }
\NormalTok{            corrientes[}\DecValTok{1}\NormalTok{] }\SpecialCharTok{{-}}\NormalTok{ corrientes[}\DecValTok{2}\NormalTok{] }\SpecialCharTok{{-}}\NormalTok{ corrientes[}\DecValTok{3}\NormalTok{]))}
\end{Highlighting}
\end{Shaded}

\begin{verbatim}
I1 - I2 - I3 = -0.000000 (≈ 0) ✓
\end{verbatim}

\begin{Shaded}
\begin{Highlighting}[]
\FunctionTok{cat}\NormalTok{(}\FunctionTok{sprintf}\NormalTok{(}\StringTok{"R1*I1 + R2*I2 = \%.4f (= \%.0f) ✓}\SpecialCharTok{\textbackslash{}n}\StringTok{"}\NormalTok{, }
\NormalTok{            R1}\SpecialCharTok{*}\NormalTok{corrientes[}\DecValTok{1}\NormalTok{] }\SpecialCharTok{+}\NormalTok{ R2}\SpecialCharTok{*}\NormalTok{corrientes[}\DecValTok{2}\NormalTok{], U))}
\end{Highlighting}
\end{Shaded}

\begin{verbatim}
R1*I1 + R2*I2 = 12.0000 (= 12) ✓
\end{verbatim}

\begin{Shaded}
\begin{Highlighting}[]
\FunctionTok{cat}\NormalTok{(}\FunctionTok{sprintf}\NormalTok{(}\StringTok{"R2*I2 {-} R3*I3 = \%.6f (≈ 0) ✓}\SpecialCharTok{\textbackslash{}n}\StringTok{"}\NormalTok{, }
\NormalTok{            R2}\SpecialCharTok{*}\NormalTok{corrientes[}\DecValTok{2}\NormalTok{] }\SpecialCharTok{{-}}\NormalTok{ R3}\SpecialCharTok{*}\NormalTok{corrientes[}\DecValTok{3}\NormalTok{]))}
\end{Highlighting}
\end{Shaded}

\begin{verbatim}
R2*I2 - R3*I3 = 0.000000 (≈ 0) ✓
\end{verbatim}

\section{Triángulo de Pascal y Binomio de
Newton}\label{triuxe1ngulo-de-pascal-y-binomio-de-newton}

\subsection{Construcción del Triángulo de
Pascal}\label{construcciuxf3n-del-triuxe1ngulo-de-pascal}

El \textbf{Triángulo de Pascal} es un esquema numérico infinito que
permite determinar directamente los \textbf{coeficientes binomiales}
\(\binom{n}{k}\).

\textbf{Reglas de Construcción}

\begin{enumerate}
\def\labelenumi{\arabic{enumi}.}
\tightlist
\item
  \textbf{Bordes laterales:} Todos son 1
\item
  \textbf{Valores internos:} Cada número es la suma de los dos números
  superiores
\item
  \textbf{Propiedad recursiva:}
  \(\binom{n}{k} + \binom{n}{k+1} = \binom{n+1}{k+1}\)
\end{enumerate}

\begin{Shaded}
\begin{Highlighting}[]
\FunctionTok{library}\NormalTok{(ggplot2)}

\CommentTok{\# Función para calcular coeficientes binomiales}
\NormalTok{generar\_pascal }\OtherTok{\textless{}{-}} \ControlFlowTok{function}\NormalTok{(n\_filas) \{}
\NormalTok{  triangulo }\OtherTok{\textless{}{-}} \FunctionTok{list}\NormalTok{()}
  \ControlFlowTok{for}\NormalTok{ (n }\ControlFlowTok{in} \DecValTok{0}\SpecialCharTok{:}\NormalTok{(n\_filas}\DecValTok{{-}1}\NormalTok{)) \{}
\NormalTok{    fila }\OtherTok{\textless{}{-}} \FunctionTok{sapply}\NormalTok{(}\DecValTok{0}\SpecialCharTok{:}\NormalTok{n, }\ControlFlowTok{function}\NormalTok{(k) }\FunctionTok{choose}\NormalTok{(n, k))}
\NormalTok{    triangulo[[n}\SpecialCharTok{+}\DecValTok{1}\NormalTok{]] }\OtherTok{\textless{}{-}}\NormalTok{ fila}
\NormalTok{  \}}
  \FunctionTok{return}\NormalTok{(triangulo)}
\NormalTok{\}}

\CommentTok{\# Generar triángulo}
\NormalTok{n\_filas }\OtherTok{\textless{}{-}} \DecValTok{10}
\NormalTok{triangulo }\OtherTok{\textless{}{-}} \FunctionTok{generar\_pascal}\NormalTok{(n\_filas)}

\CommentTok{\# Crear datos para visualización}
\NormalTok{datos\_pascal }\OtherTok{\textless{}{-}} \FunctionTok{data.frame}\NormalTok{()}
\ControlFlowTok{for}\NormalTok{ (n }\ControlFlowTok{in} \DecValTok{0}\SpecialCharTok{:}\NormalTok{(n\_filas}\DecValTok{{-}1}\NormalTok{)) \{}
\NormalTok{  fila }\OtherTok{\textless{}{-}}\NormalTok{ triangulo[[n}\SpecialCharTok{+}\DecValTok{1}\NormalTok{]]}
  \ControlFlowTok{for}\NormalTok{ (k }\ControlFlowTok{in} \DecValTok{0}\SpecialCharTok{:}\NormalTok{n) \{}
\NormalTok{    datos\_pascal }\OtherTok{\textless{}{-}} \FunctionTok{rbind}\NormalTok{(datos\_pascal, }\FunctionTok{data.frame}\NormalTok{(}
      \AttributeTok{x =}\NormalTok{ k }\SpecialCharTok{{-}}\NormalTok{ n}\SpecialCharTok{/}\DecValTok{2}\NormalTok{,}
      \AttributeTok{y =} \SpecialCharTok{{-}}\NormalTok{n,}
      \AttributeTok{valor =}\NormalTok{ fila[k}\SpecialCharTok{+}\DecValTok{1}\NormalTok{],}
      \AttributeTok{n =}\NormalTok{ n,}
      \AttributeTok{k =}\NormalTok{ k}
\NormalTok{    ))}
\NormalTok{  \}}
\NormalTok{\}}

\CommentTok{\# Visualizar}
\FunctionTok{ggplot}\NormalTok{(datos\_pascal, }\FunctionTok{aes}\NormalTok{(}\AttributeTok{x =}\NormalTok{ x, }\AttributeTok{y =}\NormalTok{ y)) }\SpecialCharTok{+}
  \FunctionTok{geom\_point}\NormalTok{(}\FunctionTok{aes}\NormalTok{(}\AttributeTok{color =} \FunctionTok{log10}\NormalTok{(valor }\SpecialCharTok{+} \DecValTok{1}\NormalTok{)), }\AttributeTok{size =} \DecValTok{8}\NormalTok{) }\SpecialCharTok{+}
  \FunctionTok{geom\_text}\NormalTok{(}\FunctionTok{aes}\NormalTok{(}\AttributeTok{label =}\NormalTok{ valor), }\AttributeTok{color =} \StringTok{"white"}\NormalTok{, }\AttributeTok{fontface =} \StringTok{"bold"}\NormalTok{, }\AttributeTok{size =} \DecValTok{3}\NormalTok{) }\SpecialCharTok{+}
  \FunctionTok{scale\_color\_gradient}\NormalTok{(}\AttributeTok{low =} \StringTok{"\#326891"}\NormalTok{, }\AttributeTok{high =} \StringTok{"\#cc0000"}\NormalTok{) }\SpecialCharTok{+}
  \FunctionTok{theme\_void}\NormalTok{() }\SpecialCharTok{+}
  \FunctionTok{theme}\NormalTok{(}\AttributeTok{legend.position =} \StringTok{"none"}\NormalTok{,}
        \AttributeTok{plot.title =} \FunctionTok{element\_text}\NormalTok{(}\AttributeTok{hjust =} \FloatTok{0.5}\NormalTok{, }\AttributeTok{face =} \StringTok{"bold"}\NormalTok{, }\AttributeTok{size =} \DecValTok{14}\NormalTok{)) }\SpecialCharTok{+}
  \FunctionTok{labs}\NormalTok{(}\AttributeTok{title =} \StringTok{"Triángulo de Pascal"}\NormalTok{) }\SpecialCharTok{+}
  \FunctionTok{coord\_fixed}\NormalTok{()}
\end{Highlighting}
\end{Shaded}

\begin{figure}[H]

{\centering \includegraphics{notas_files/figure-pdf/triangulo-pascal-1.pdf}

}

\caption{Triángulo de Pascal (primeras 10 filas)}

\end{figure}%

\subsection{Binomio de Newton}\label{binomio-de-newton}

El triángulo de Pascal proporciona los coeficientes para la expansión de
potencias de binomios:

\textbf{Fórmula del Binomio de Newton}

\((a + b)^n = \sum_{k=0}^{n} \binom{n}{k} a^{n-k} b^k\)

donde \(\binom{n}{k}\) se lee ``\(n\) sobre \(k\)'' y se calcula como:
\(\binom{n}{k} = \frac{n!}{k!(n-k)!}\)

\textbf{Ejemplos de Desarrollo}

\textbf{Para \(n = 2\):} Coeficientes 1, 2, 1
\((a + b)^2 = a^2 + 2ab + b^2\)

\textbf{Para \(n = 3\):} Coeficientes 1, 3, 3, 1
\((a + b)^3 = a^3 + 3a^2b + 3ab^2 + b^3\)

\textbf{Para \(n = 4\):} Coeficientes 1, 4, 6, 4, 1
\((a + b)^4 = a^4 + 4a^3b + 6a^2b^2 + 4ab^3 + b^4\)

\begin{Shaded}
\begin{Highlighting}[]
\CommentTok{\# Función para expandir (a + b)\^{}n simbólicamente}
\NormalTok{expandir\_binomio }\OtherTok{\textless{}{-}} \ControlFlowTok{function}\NormalTok{(n) \{}
\NormalTok{  coef }\OtherTok{\textless{}{-}} \FunctionTok{choose}\NormalTok{(n, }\DecValTok{0}\SpecialCharTok{:}\NormalTok{n)}
  \FunctionTok{cat}\NormalTok{(}\FunctionTok{sprintf}\NormalTok{(}\StringTok{"(a + b)\^{}\%d = "}\NormalTok{, n))}
  
\NormalTok{  terminos }\OtherTok{\textless{}{-}} \FunctionTok{character}\NormalTok{()}
  \ControlFlowTok{for}\NormalTok{ (k }\ControlFlowTok{in} \DecValTok{0}\SpecialCharTok{:}\NormalTok{n) \{}
    \CommentTok{\# Coeficiente}
    \ControlFlowTok{if}\NormalTok{ (coef[k}\SpecialCharTok{+}\DecValTok{1}\NormalTok{] }\SpecialCharTok{==} \DecValTok{1} \SpecialCharTok{\&\&}\NormalTok{ k }\SpecialCharTok{\textgreater{}} \DecValTok{0} \SpecialCharTok{\&\&}\NormalTok{ k }\SpecialCharTok{\textless{}}\NormalTok{ n) \{}
\NormalTok{      coef\_str }\OtherTok{\textless{}{-}} \StringTok{""}
\NormalTok{    \} }\ControlFlowTok{else} \ControlFlowTok{if}\NormalTok{ (coef[k}\SpecialCharTok{+}\DecValTok{1}\NormalTok{] }\SpecialCharTok{==} \DecValTok{1}\NormalTok{) \{}
\NormalTok{      coef\_str }\OtherTok{\textless{}{-}} \ControlFlowTok{if}\NormalTok{ (k }\SpecialCharTok{==} \DecValTok{0} \SpecialCharTok{||}\NormalTok{ k }\SpecialCharTok{==}\NormalTok{ n) }\StringTok{"1"} \ControlFlowTok{else} \StringTok{""}
\NormalTok{    \} }\ControlFlowTok{else}\NormalTok{ \{}
\NormalTok{      coef\_str }\OtherTok{\textless{}{-}} \FunctionTok{as.character}\NormalTok{(coef[k}\SpecialCharTok{+}\DecValTok{1}\NormalTok{])}
\NormalTok{    \}}
    
    \CommentTok{\# Potencia de a}
\NormalTok{    pot\_a }\OtherTok{\textless{}{-}}\NormalTok{ n }\SpecialCharTok{{-}}\NormalTok{ k}
\NormalTok{    a\_str }\OtherTok{\textless{}{-}} \ControlFlowTok{if}\NormalTok{ (pot\_a }\SpecialCharTok{==} \DecValTok{0}\NormalTok{) }\StringTok{""} \ControlFlowTok{else} \ControlFlowTok{if}\NormalTok{ (pot\_a }\SpecialCharTok{==} \DecValTok{1}\NormalTok{) }\StringTok{"a"} \ControlFlowTok{else} \FunctionTok{sprintf}\NormalTok{(}\StringTok{"a\^{}\%d"}\NormalTok{, pot\_a)}
    
    \CommentTok{\# Potencia de b}
\NormalTok{    pot\_b }\OtherTok{\textless{}{-}}\NormalTok{ k}
\NormalTok{    b\_str }\OtherTok{\textless{}{-}} \ControlFlowTok{if}\NormalTok{ (pot\_b }\SpecialCharTok{==} \DecValTok{0}\NormalTok{) }\StringTok{""} \ControlFlowTok{else} \ControlFlowTok{if}\NormalTok{ (pot\_b }\SpecialCharTok{==} \DecValTok{1}\NormalTok{) }\StringTok{"b"} \ControlFlowTok{else} \FunctionTok{sprintf}\NormalTok{(}\StringTok{"b\^{}\%d"}\NormalTok{, pot\_b)}
    
    \CommentTok{\# Combinar}
\NormalTok{    termino }\OtherTok{\textless{}{-}} \FunctionTok{paste0}\NormalTok{(coef\_str, a\_str, b\_str)}
    \ControlFlowTok{if}\NormalTok{ (termino }\SpecialCharTok{==} \StringTok{""}\NormalTok{) termino }\OtherTok{\textless{}{-}} \StringTok{"1"}
\NormalTok{    terminos }\OtherTok{\textless{}{-}} \FunctionTok{c}\NormalTok{(terminos, termino)}
\NormalTok{  \}}
  
  \FunctionTok{cat}\NormalTok{(}\FunctionTok{paste}\NormalTok{(terminos, }\AttributeTok{collapse =} \StringTok{" + "}\NormalTok{), }\StringTok{"}\SpecialCharTok{\textbackslash{}n\textbackslash{}n}\StringTok{"}\NormalTok{)}
\NormalTok{\}}

\CommentTok{\# Ejemplos}
\ControlFlowTok{for}\NormalTok{ (n }\ControlFlowTok{in} \DecValTok{2}\SpecialCharTok{:}\DecValTok{6}\NormalTok{) \{}
  \FunctionTok{expandir\_binomio}\NormalTok{(n)}
\NormalTok{\}}
\end{Highlighting}
\end{Shaded}

\begin{verbatim}
(a + b)^2 = 1a^2 + 2ab + 1b^2 

(a + b)^3 = 1a^3 + 3a^2b + 3ab^2 + 1b^3 

(a + b)^4 = 1a^4 + 4a^3b + 6a^2b^2 + 4ab^3 + 1b^4 

(a + b)^5 = 1a^5 + 5a^4b + 10a^3b^2 + 10a^2b^3 + 5ab^4 + 1b^5 

(a + b)^6 = 1a^6 + 6a^5b + 15a^4b^2 + 20a^3b^3 + 15a^2b^4 + 6ab^5 + 1b^6 
\end{verbatim}

\section{Método de Newton para Ecuaciones No
Lineales}\label{muxe9todo-de-newton-para-ecuaciones-no-lineales}

Para ecuaciones trascendentes o complejas sin solución analítica, el
\textbf{Método de Newton} (\emph{Tangentenverfahren}) proporciona
aproximaciones iterativas de alta precisión.

\textbf{Fórmula Iterativa de Newton}

\(x_{n+1} = x_n - \frac{f(x_n)}{f'(x_n)}\)

\textbf{Características:} - Convergencia cuadrática cerca de la raíz -
Requiere calcular la derivada \(f'(x)\) - Sensible a la elección del
punto inicial \(x_0\)

\begin{Shaded}
\begin{Highlighting}[]
\FunctionTok{library}\NormalTok{(ggplot2)}

\CommentTok{\# Función y su derivada}
\NormalTok{f }\OtherTok{\textless{}{-}} \ControlFlowTok{function}\NormalTok{(x) x}\SpecialCharTok{\^{}}\DecValTok{2} \SpecialCharTok{{-}} \DecValTok{2}
\NormalTok{f\_prima }\OtherTok{\textless{}{-}} \ControlFlowTok{function}\NormalTok{(x) }\DecValTok{2}\SpecialCharTok{*}\NormalTok{x}

\CommentTok{\# Método de Newton}
\NormalTok{newton }\OtherTok{\textless{}{-}} \ControlFlowTok{function}\NormalTok{(x0, f, f\_prima, }\AttributeTok{tol =} \FloatTok{1e{-}10}\NormalTok{, }\AttributeTok{max\_iter =} \DecValTok{20}\NormalTok{) \{}
\NormalTok{  x }\OtherTok{\textless{}{-}}\NormalTok{ x0}
\NormalTok{  iteraciones }\OtherTok{\textless{}{-}} \FunctionTok{data.frame}\NormalTok{(}\AttributeTok{n =} \DecValTok{0}\NormalTok{, }\AttributeTok{x =}\NormalTok{ x0, }\AttributeTok{fx =} \FunctionTok{f}\NormalTok{(x0))}
  
  \ControlFlowTok{for}\NormalTok{ (i }\ControlFlowTok{in} \DecValTok{1}\SpecialCharTok{:}\NormalTok{max\_iter) \{}
\NormalTok{    x\_nuevo }\OtherTok{\textless{}{-}}\NormalTok{ x }\SpecialCharTok{{-}} \FunctionTok{f}\NormalTok{(x) }\SpecialCharTok{/} \FunctionTok{f\_prima}\NormalTok{(x)}
\NormalTok{    iteraciones }\OtherTok{\textless{}{-}} \FunctionTok{rbind}\NormalTok{(iteraciones, }\FunctionTok{data.frame}\NormalTok{(}
      \AttributeTok{n =}\NormalTok{ i, }
      \AttributeTok{x =}\NormalTok{ x\_nuevo, }
      \AttributeTok{fx =} \FunctionTok{f}\NormalTok{(x\_nuevo)}
\NormalTok{    ))}
    
    \ControlFlowTok{if}\NormalTok{ (}\FunctionTok{abs}\NormalTok{(x\_nuevo }\SpecialCharTok{{-}}\NormalTok{ x) }\SpecialCharTok{\textless{}}\NormalTok{ tol) }\ControlFlowTok{break}
\NormalTok{    x }\OtherTok{\textless{}{-}}\NormalTok{ x\_nuevo}
\NormalTok{  \}}
  
  \FunctionTok{return}\NormalTok{(iteraciones)}
\NormalTok{\}}

\CommentTok{\# Ejecutar Newton con x0 = 2}
\NormalTok{resultado }\OtherTok{\textless{}{-}} \FunctionTok{newton}\NormalTok{(}\AttributeTok{x0 =} \DecValTok{2}\NormalTok{, f, f\_prima)}

\FunctionTok{cat}\NormalTok{(}\StringTok{"Método de Newton para encontrar √2}\SpecialCharTok{\textbackslash{}n}\StringTok{"}\NormalTok{)}
\end{Highlighting}
\end{Shaded}

\begin{verbatim}
Método de Newton para encontrar √2
\end{verbatim}

\begin{Shaded}
\begin{Highlighting}[]
\FunctionTok{cat}\NormalTok{(}\StringTok{"Ecuación: f(x) = x² {-} 2 = 0}\SpecialCharTok{\textbackslash{}n\textbackslash{}n}\StringTok{"}\NormalTok{)}
\end{Highlighting}
\end{Shaded}

\begin{verbatim}
Ecuación: f(x) = x² - 2 = 0
\end{verbatim}

\begin{Shaded}
\begin{Highlighting}[]
\FunctionTok{print}\NormalTok{(resultado, }\AttributeTok{digits =} \DecValTok{10}\NormalTok{)}
\end{Highlighting}
\end{Shaded}

\begin{verbatim}
  n           x              fx
1 0 2.000000000 2.000000000e+00
2 1 1.500000000 2.500000000e-01
3 2 1.416666667 6.944444444e-03
4 3 1.414215686 6.007304883e-06
5 4 1.414213562 4.510614104e-12
6 5 1.414213562 4.440892099e-16
\end{verbatim}

\begin{Shaded}
\begin{Highlighting}[]
\CommentTok{\# Visualización}
\NormalTok{x\_vals }\OtherTok{\textless{}{-}} \FunctionTok{seq}\NormalTok{(}\FloatTok{0.5}\NormalTok{, }\DecValTok{3}\NormalTok{, }\AttributeTok{by =} \FloatTok{0.01}\NormalTok{)}
\NormalTok{datos\_func }\OtherTok{\textless{}{-}} \FunctionTok{data.frame}\NormalTok{(}\AttributeTok{x =}\NormalTok{ x\_vals, }\AttributeTok{y =} \FunctionTok{f}\NormalTok{(x\_vals))}

\CommentTok{\# Crear gráfico}
\NormalTok{p }\OtherTok{\textless{}{-}} \FunctionTok{ggplot}\NormalTok{() }\SpecialCharTok{+}
  \FunctionTok{geom\_line}\NormalTok{(}\AttributeTok{data =}\NormalTok{ datos\_func, }\FunctionTok{aes}\NormalTok{(}\AttributeTok{x =}\NormalTok{ x, }\AttributeTok{y =}\NormalTok{ y), }
            \AttributeTok{color =} \StringTok{"\#326891"}\NormalTok{, }\AttributeTok{size =} \FloatTok{1.2}\NormalTok{) }\SpecialCharTok{+}
  \FunctionTok{geom\_hline}\NormalTok{(}\AttributeTok{yintercept =} \DecValTok{0}\NormalTok{, }\AttributeTok{linetype =} \StringTok{"dashed"}\NormalTok{, }\AttributeTok{alpha =} \FloatTok{0.5}\NormalTok{) }\SpecialCharTok{+}
  \FunctionTok{geom\_vline}\NormalTok{(}\AttributeTok{xintercept =} \FunctionTok{sqrt}\NormalTok{(}\DecValTok{2}\NormalTok{), }\AttributeTok{linetype =} \StringTok{"dotted"}\NormalTok{, }
             \AttributeTok{color =} \StringTok{"\#008060"}\NormalTok{, }\AttributeTok{size =} \DecValTok{1}\NormalTok{) }\SpecialCharTok{+}
  \FunctionTok{theme\_minimal}\NormalTok{() }\SpecialCharTok{+}
  \FunctionTok{labs}\NormalTok{(}\AttributeTok{x =} \StringTok{"x"}\NormalTok{, }\AttributeTok{y =} \StringTok{"f(x)"}\NormalTok{, }
       \AttributeTok{title =} \StringTok{"Método de Newton: f(x) = x² {-} 2"}\NormalTok{)}

\CommentTok{\# Añadir iteraciones}
\ControlFlowTok{for}\NormalTok{ (i }\ControlFlowTok{in} \DecValTok{1}\SpecialCharTok{:}\FunctionTok{min}\NormalTok{(}\DecValTok{4}\NormalTok{, }\FunctionTok{nrow}\NormalTok{(resultado))) \{}
\NormalTok{  xi }\OtherTok{\textless{}{-}}\NormalTok{ resultado}\SpecialCharTok{$}\NormalTok{x[i]}
\NormalTok{  yi }\OtherTok{\textless{}{-}}\NormalTok{ resultado}\SpecialCharTok{$}\NormalTok{fx[i]}
  
  \CommentTok{\# Punto}
\NormalTok{  p }\OtherTok{\textless{}{-}}\NormalTok{ p }\SpecialCharTok{+} \FunctionTok{geom\_point}\NormalTok{(}\FunctionTok{aes}\NormalTok{(}\AttributeTok{x =}\NormalTok{ xi, }\AttributeTok{y =}\NormalTok{ yi), }
                      \AttributeTok{color =} \StringTok{"\#cc0000"}\NormalTok{, }\AttributeTok{size =} \DecValTok{3}\NormalTok{)}
  
  \CommentTok{\# Tangente (solo primeras 3 iteraciones)}
  \ControlFlowTok{if}\NormalTok{ (i }\SpecialCharTok{\textless{}=} \DecValTok{3}\NormalTok{) \{}
\NormalTok{    m }\OtherTok{\textless{}{-}} \FunctionTok{f\_prima}\NormalTok{(xi)}
\NormalTok{    x\_tang }\OtherTok{\textless{}{-}} \FunctionTok{seq}\NormalTok{(}\FunctionTok{max}\NormalTok{(}\FloatTok{0.5}\NormalTok{, xi }\SpecialCharTok{{-}} \DecValTok{1}\NormalTok{), }\FunctionTok{min}\NormalTok{(}\DecValTok{3}\NormalTok{, xi }\SpecialCharTok{+} \DecValTok{1}\NormalTok{), }\AttributeTok{length.out =} \DecValTok{50}\NormalTok{)}
\NormalTok{    y\_tang }\OtherTok{\textless{}{-}}\NormalTok{ yi }\SpecialCharTok{+}\NormalTok{ m }\SpecialCharTok{*}\NormalTok{ (x\_tang }\SpecialCharTok{{-}}\NormalTok{ xi)}
    
\NormalTok{    p }\OtherTok{\textless{}{-}}\NormalTok{ p }\SpecialCharTok{+} \FunctionTok{geom\_line}\NormalTok{(}\FunctionTok{aes}\NormalTok{(}\AttributeTok{x =}\NormalTok{ x\_tang, }\AttributeTok{y =}\NormalTok{ y\_tang), }
                       \AttributeTok{color =} \StringTok{"\#e67e22"}\NormalTok{, }\AttributeTok{linetype =} \StringTok{"dashed"}\NormalTok{, }
                       \AttributeTok{alpha =} \FloatTok{0.7}\NormalTok{)}
\NormalTok{  \}}
\NormalTok{\}}

\FunctionTok{print}\NormalTok{(p)}
\end{Highlighting}
\end{Shaded}

\begin{figure}[H]

{\centering \includegraphics{notas_files/figure-pdf/metodo-newton-1.pdf}

}

\caption{Convergencia del Método de Newton para f(x) = x² - 2}

\end{figure}%

\begin{Shaded}
\begin{Highlighting}[]
\CommentTok{\# Probar diferentes puntos iniciales}
\NormalTok{puntos\_iniciales }\OtherTok{\textless{}{-}} \FunctionTok{c}\NormalTok{(}\FloatTok{0.5}\NormalTok{, }\DecValTok{1}\NormalTok{, }\DecValTok{2}\NormalTok{, }\DecValTok{3}\NormalTok{)}
\NormalTok{resultados\_todos }\OtherTok{\textless{}{-}} \FunctionTok{list}\NormalTok{()}

\ControlFlowTok{for}\NormalTok{ (x0 }\ControlFlowTok{in}\NormalTok{ puntos\_iniciales) \{}
\NormalTok{  res }\OtherTok{\textless{}{-}} \FunctionTok{newton}\NormalTok{(x0, f, f\_prima, }\AttributeTok{max\_iter =} \DecValTok{10}\NormalTok{)}
\NormalTok{  res}\SpecialCharTok{$}\NormalTok{x0 }\OtherTok{\textless{}{-}}\NormalTok{ x0}
\NormalTok{  resultados\_todos[[}\FunctionTok{length}\NormalTok{(resultados\_todos) }\SpecialCharTok{+} \DecValTok{1}\NormalTok{]] }\OtherTok{\textless{}{-}}\NormalTok{ res}
\NormalTok{\}}

\NormalTok{datos\_conv }\OtherTok{\textless{}{-}} \FunctionTok{do.call}\NormalTok{(rbind, resultados\_todos)}

\FunctionTok{ggplot}\NormalTok{(datos\_conv, }\FunctionTok{aes}\NormalTok{(}\AttributeTok{x =}\NormalTok{ n, }\AttributeTok{y =} \FunctionTok{abs}\NormalTok{(x }\SpecialCharTok{{-}} \FunctionTok{sqrt}\NormalTok{(}\DecValTok{2}\NormalTok{)), }\AttributeTok{color =} \FunctionTok{factor}\NormalTok{(x0))) }\SpecialCharTok{+}
  \FunctionTok{geom\_line}\NormalTok{(}\AttributeTok{size =} \FloatTok{1.2}\NormalTok{) }\SpecialCharTok{+}
  \FunctionTok{geom\_point}\NormalTok{(}\AttributeTok{size =} \FloatTok{2.5}\NormalTok{) }\SpecialCharTok{+}
  \FunctionTok{scale\_y\_log10}\NormalTok{() }\SpecialCharTok{+}
  \FunctionTok{scale\_color\_manual}\NormalTok{(}\AttributeTok{values =} \FunctionTok{c}\NormalTok{(}\StringTok{"\#326891"}\NormalTok{, }\StringTok{"\#cc0000"}\NormalTok{, }\StringTok{"\#008060"}\NormalTok{, }\StringTok{"\#e67e22"}\NormalTok{),}
                     \AttributeTok{name =} \StringTok{"Punto inicial x₀"}\NormalTok{) }\SpecialCharTok{+}
  \FunctionTok{theme\_minimal}\NormalTok{() }\SpecialCharTok{+}
  \FunctionTok{theme}\NormalTok{(}\AttributeTok{legend.position =} \StringTok{"top"}\NormalTok{) }\SpecialCharTok{+}
  \FunctionTok{labs}\NormalTok{(}\AttributeTok{x =} \StringTok{"Iteración n"}\NormalTok{, }
       \AttributeTok{y =} \StringTok{"|xₙ {-} √2| (escala logarítmica)"}\NormalTok{,}
       \AttributeTok{title =} \StringTok{"Convergencia del Método de Newton"}\NormalTok{,}
       \AttributeTok{subtitle =} \StringTok{"Velocidad de convergencia según punto inicial"}\NormalTok{)}
\end{Highlighting}
\end{Shaded}

\begin{verbatim}
Warning in scale_y_log10(): log-10 transformation introduced infinite values.
log-10 transformation introduced infinite values.
\end{verbatim}

\begin{figure}[H]

{\centering \includegraphics{notas_files/figure-pdf/newton-comparacion-1.pdf}

}

\caption{Comparación de convergencia según punto inicial}

\end{figure}%

\section{Conclusiones y Síntesis}\label{conclusiones-y-suxedntesis}

\subsection{Resumen Estructural}\label{resumen-estructural}

Los fundamentos matemáticos presentados en este documento constituyen el
\textbf{andamiaje conceptual} sobre el cual se erigen todas las
disciplinas científico-técnicas modernas:

\begin{tcolorbox}[enhanced jigsaw, leftrule=.75mm, title=\textcolor{quarto-callout-tip-color}{\faLightbulb}\hspace{0.5em}{Jerarquía Conceptual}, colback=white, breakable, coltitle=black, bottomrule=.15mm, colframe=quarto-callout-tip-color-frame, arc=.35mm, toprule=.15mm, opacitybacktitle=0.6, rightrule=.15mm, opacityback=0, bottomtitle=1mm, toptitle=1mm, colbacktitle=quarto-callout-tip-color!10!white, titlerule=0mm, left=2mm]

\begin{enumerate}
\def\labelenumi{\arabic{enumi}.}
\tightlist
\item
  \textbf{Teoría de Conjuntos} → Proporciona el lenguaje lógico
  fundamental
\item
  \textbf{Números Reales} → Establecen la métrica y el continuo
\item
  \textbf{Ecuaciones} → Mecanismo de modelado matemático
\item
  \textbf{Sistemas Lineales} → Estructura para problemas multivariables
\item
  \textbf{Métodos Numéricos} → Herramientas computacionales prácticas
\end{enumerate}

\end{tcolorbox}

\subsection{Principios Fundamentales para
Ingenieros}\label{principios-fundamentales-para-ingenieros}

De acuerdo con Papula (2024), el rigor matemático en ingeniería se
fundamenta en:

\begin{itemize}
\tightlist
\item
  \textbf{Precisión lógica:} Toda afirmación debe ser verificable
\item
  \textbf{Coherencia estructural:} Los sistemas deben ser consistentes
\item
  \textbf{Aplicabilidad práctica:} Los conceptos deben resolver
  problemas reales
\item
  \textbf{Generalización:} Las soluciones particulares revelan patrones
  universales
\end{itemize}

\subsection{Próximos Temas}\label{pruxf3ximos-temas}

En volúmenes posteriores se abordarán:

\begin{itemize}
\tightlist
\item
  Álgebra vectorial y geometría analítica
\item
  Funciones y límites
\item
  Cálculo diferencial e integral
\item
  Ecuaciones diferenciales
\item
  Series y transformadas
\end{itemize}

\begin{center}\rule{0.5\linewidth}{0.5pt}\end{center}

\section{Referencias}\label{referencias}

\phantomsection\label{refs}
\begin{CSLReferences}{1}{0}
\bibitem[\citeproctext]{ref-papula2024}
Papula, L. (2024). \emph{Mathematik für Ingenieure und
Naturwissenschafter, Band 1: Ein Lehr- und Arbeitsbuch für das
Grundstudium} (16., überarbeitete und erweiterte Auflage). Springer
Vieweg. \url{https://doi.org/10.1007/978-3-658-45802-7}

\end{CSLReferences}

\begin{center}\rule{0.5\linewidth}{0.5pt}\end{center}

::: \{.callout-note icon=false\} \#\# 📚 Información del Documento

\textbf{Fuente Principal:} Papula, L. (2024). \emph{Mathematik für
Ingenieure und Naturwissenschaftler, Band 1} (16.ª edición). Springer
Vieweg.\\
\textbf{ISBN:} 978-3-658-45801-0\\
\textbf{DOI:}
\href{https://doi.org/10.1007/978-3-658-45802-7}{10.1007/978-3-658-45802-7}

\textbf{Autor del Documento:} Emanuel Quintana Silva\\
\textbf{Afiliación:} Universidad Pedagógica y Tecnológica de Colombia
(UPTC)\\
\textbf{Especialización:} Econometría Computacional \textbar{}
Aplicaciones R/Python\\
\textbf{Contacto:} emanuel.quintana@uptc.edu.co\\
\textbf{ORCID:}
\href{https://orcid.org/0009-0006-8419-2805}{0009-0006-8419-2805}

\begin{center}\rule{0.5\linewidth}{0.5pt}\end{center}

\textbf{THE SCIENCE TIMES} \textbar{} MATHEMATISCHE GRUNDLAGEN
\textbar{} ENERO 2026



\end{document}
